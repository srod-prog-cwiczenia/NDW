\documentclass{amsart}

\usepackage[mathscr]{euscript}
\usepackage{amssymb}
\usepackage{amsmath}
\usepackage{latexsym}

\makeatletter
\renewcommand\@biblabel[1]{#1.}
\makeatother
\newtheorem{thm}{Theorem}[section]
\newtheorem{lem}[thm]{Lemma}
\newtheorem{prop}[thm]{Proposition}
\newtheorem{fact}[thm]{Fact}
\newtheorem{cor}[thm]{Corollary}
\theoremstyle{definition}
\newtheorem{problem}[thm]{Problem}
\newtheorem{df}[thm]{Definition}
\newtheorem{remark}[thm]{Remark}
\newtheorem{ex}[thm]{Example}
\newtheorem{conj}[thm]{Conjecture}

\newcommand{\N}{{\mathbb N}}
\newcommand{\Z}{{\mathbb Z}}
\newcommand{\R}{{\mathbb R}}
\newcommand{\Q}{{\mathbb Q}}
\newcommand{\Fin}{\textrm{Fin}}
\newcommand{\I}{\mathcal I}
\newcommand{\J}{\mathcal J}
\newcommand{\T}{\mathcal{T}}
\newcommand{\B}{\mathcal{B}}
\newcommand{\SqrFr}{\mathbb{SF}}
\newcommand{\calF}{\mathcal{F}}
\newcommand{\modulo}{\textrm{mod }}
\newcommand{\InfSubs}{[\N]^{\omega}}
\newcommand{\finbw}{\text{FinBW}}
\newcommand{\FinPart}{(\N)^{< \omega}}
\newcommand{\MB}{S^0}
\newcommand{\MBC}{\mathcal{MBC}}
\newcommand{\Seg}{\mathrm{Seg}}
\newcommand{\NULL}{\mathrm{NULL}}
\newcommand{\NWD}{\mathrm{NWD}}
\newcommand{\INULL}{\I_\mathrm{NULL}}			  
\newcommand{\cl}{\mathrm{cl}}
\newcommand{\interior}{\mathrm{int}}
%%%@@@ stara wersja oznaczenia, a ponizej nowa:
%%%\newcommand{\arithseq}[2]{\{#1n + #2\}}
\newcommand{\arithseq}[2]{\langle#2, #1\rangle}

\title[Ideals of nowhere dense sets in some topologies on positive integers]{Ideals of nowhere dense sets in some topologies on positive integers}

\author{Marta Kwela}
\address{Marta Kwela, Institute of Mathematics, Faculty of Mathematics, Physics and Informatics, University of Gda\'{n}sk, ul.~Wita Stwosza 57, 80-308 Gda\'{n}sk, Poland}
\email{Marta.Kwela@mat.ug.edu.pl}

\author{Andrzej Nowik}
\address{Andrzej Nowik, Institute of Mathematics, Faculty of Mathematics, Physics and Informatics, University of Gda\'{n}sk, ul.~Wita Stwosza 57, 80-308 Gda\'{n}sk, Poland}
\email{andrzej@mat.ug.edu.pl}

\begin{document}
\begin{abstract}
We investigate the ideals of nowhere dense sets in three topologies on $\N$ (namely, the Furstenberg's, Golomb's, and Kirch's topology) related to arithmetic progressions. In particular, we explore relationships between these ideals, and show that each of them has a topological representation and cannot be extended to a summable ideal. Moreover, we study a related notion of Marczewski-Burstin countable representability.
\end{abstract}
\maketitle


\section{Introduction}

Let $\N$ denote the set of positive integers and $\N_0$ -- the set of non-negative integers. Following \cite{K} for all $a,b\in\N$ the symbol $\arithseq{a}{b}$ stands for the infinite arithmetic progression with the initial term $b$ and the difference $a$:
$$\arithseq{a}{b} = \{an+b :\ n\in\N_0\} = \{b,\ b+a,\ b+2a,\ \ldots\}. $$
We use the symbol $(a,b)$ to denote the greatest common divisor of $a$ and $b$. 
The letter $\mathbb{P}$ symbolizes the set of all prime numbers and $\Theta(a)$ stands for the set of all prime factors of $a\in\N$.
By $\mathbb{SF}$ let us denote the set of \emph{square-free numbers} (i.e., numbers not divisible by any square greater than 1):
$$\SqrFr = \{1,2,3,5,6,7,10,11,\ldots\}.$$
By \emph{squareful numbers} we mean numbers which are not square-free, i.e., numbers for which an exponent of some prime factor is at least 2.

Let $\InfSubs$ denote the family of all infinite subsets of $\N$. 

By treating the power set $\mathcal{P}(\N)$ as the space $2^\N$ of all functions $f\colon\N\to 2$ (equipped with the product topology, where each space $2= \left\{0,1\right\}$ carries the discrete topology) and identifying subsets of $\N$ with their characteristic functions, we can talk about descriptive complexity of subsets of $\mathcal{P}(\N)$.

For other basic notions concerning set theory and topology see, e.g., \cite{Kechris}.


\section*{Three topologies}

One can consider three topologies on $\N$:
\begin{itemize}
\item \emph{Furstenberg's topology} $\T_F$ \\
			with the base $\B_F = \{\arithseq{a}{b} :\ b\leq a\}$,
\item \emph{Golomb's topology} $\T_G$ \\
			with the base $\B_G = \{\arithseq{a}{b} :\ (a,b)=1,\ b<a\}$,
\item \emph{Kirch's topology} $\T_K$ \\
			with the base $\B_K = \{\arithseq{a}{b} :\ (a,b)=1,\ b<a,\ a\in\SqrFr\}$.
\end{itemize}

The topology $\T_F$ was introduced in 1955 by H. Furstenberg in \cite{F}. With its use he presented an elegant topological proof of the existence of infinitely many prime numbers. In 1959, S. Golomb in \cite{G} presented a similar proof using the topology $\T_G$ defined in 1953 by M. Brown in \cite{B}. In 1969, A. Kirch in \cite{K} defined the topology $\T_K$, weaker than the topology of Golomb. All of these topologies have recently been studied by P. Szczuka, e.g., in \cite{Szczuka1}, \cite{Szczuka2}, \cite{Szczuka3}.

Actually, the Furstenberg's topology was originally defined on $\Z$, with the base consisting of all doubly infinite arithmetic progressions (from $-\infty$ to $+\infty$). It turned $\Z$ into a metrizable, zero-dimensional, and totally disconnected space. In this paper, in order to make our considerations more unified, we trim this topology to $\N$. Note that the main properties are preserved: being a Hausdorff, regular, and totally disconnected space is hereditary. $(\N,\T_F)$ also remains second-countable and thus, from the Tychonoff-Urysohn metrization theorem, we get that the space is metrizable. The requirement that $b\leq a$ guarantees that every basic set is closed, so the space is still zero-dimensional.

The topologies of Golomb and Kirch both are Hausdorff but not regular, and connected -- however, $\T_G$ is not locally connected, as opposed to $\T_K$.


\section*{Three ideals}

An \emph{ideal} on $\N$ is a family of subsets of $\N$, closed under taking finite unions and subsets of its elements. We assume that an ideal is proper ($\neq \mathcal{P}(\N)$) and contains all finite sets. By $\Fin$ we denote the ideal of all finite subsets of $\N$.

%%%@@@ podmienilem na razie tylko slowo "decent" na "non-trivial"
Obviously, in any (non-trivial\footnote{\ The topology should have a base consisting only of 
infinite sets -- otherwise, the ideal of nowhere dense sets would not contain all finite sets.}) topology, the nowhere dense sets form an ideal. Let us then define three ideals on $\N$:
\begin{itemize}
\item \emph{Furstenberg's ideal} $\I_F$ of all nowhere dense sets in $\T_F$,
\item \emph{Golomb's ideal} $\I_G$ of all nowhere dense sets in $\T_G$,
\item \emph{Kirch's ideal} $\I_K$ of all nowhere dense sets in $\T_K$.
\end{itemize}


\section*{Splitting property}

Let us say that a family $\mathcal{F} \subseteq \InfSubs$ has the \emph{splitting property} if for any $F \in \mathcal{F}$ one can find $F_1, F_2 \in \mathcal{F}$ such that $F_1 \cup F_2 \subseteq F$ and $F_1 \cap F_2 = \emptyset$. Notice that if a family $\mathcal{F}$ has the splitting property, then it also has the \emph{countable splitting property}, i.e., for any $F \in \mathcal{F}$ there exists an infinite countable family $\mathcal{G} \subseteq \mathcal{F}$ such that $\bigcup{\mathcal{G}} \subseteq F$ and the family $\mathcal{G}$ is pairwise disjoint.

%%%%@@@ z uwagi na sugestie Recenzenta na razie (poki co) biore to
%%% w komentarz, alisci jednak nalezy to przeanalizowac czy jednak mozna
%%% to w jakos sposob wplesc w tresc pracy zadoscuczyniajac jednoczesnie Recenzentowi
%\begin{prop}
%The families $\B_F$, $\B_G$, and $\B_K$ have the splitting property.
%\end{prop}
%
%\begin{proof}
%For the case of the Furstenberg's topology it suffices to observe that for any $\arithseq{a}{b}\in\B_F$ we have: $\arithseq{2a}{b} \subseteq \arithseq{a}{b}$, $\arithseq{2an}{a + b} \subseteq \arithseq{a}{b}$, and $\arithseq{2a}{b} \cap \arithseq{2a}{a + b} = \emptyset$. Moreover, we assume that $b\leq a$, so $\arithseq{2a}{b}\in\B_F$ as $b\leq a \leq 2a$, and $\{2an + a + b\}\in\B_F$ as $a+b\leq a+a = 2a$.
%
%\begin{center}
%\begin{picture}(260,180)
%\put(0,0){\makebox(0,0){$\vdots$}}
%\put(0,10){\makebox(0,0){$\{16an+b\}$}}
%\put(100,10){\makebox(0,0){$\{16an+8a+b\}$}}
%\put(50,50){\makebox(0,0){$\{8an+b\}$}}
%\put(65,40){\vector(1,-1){20}}
%\put(35,40){\vector(-1,-1){20}}
%\put(150,50){\makebox(0,0){$\{8an+4a+b\}$}}
%\put(100,90){\makebox(0,0){$\{4an+b\}$}}
%\put(115,80){\vector(1,-1){20}}
%\put(85,80){\vector(-1,-1){20}}
%\put(200,90){\makebox(0,0){$\{4an+2a+b\}$}}
%\put(150,130){\makebox(0,0){$\{2an+b\}$}}
%\put(165,120){\vector(1,-1){20}}
%\put(135,120){\vector(-1,-1){20}}
%\put(250,130){\makebox(0,0){$\{2an+a+b\}$}}
%\put(200,170){\makebox(0,0){$\{an+b\}$}}
%\put(215,160){\vector(1,-1){20}}
%\put(185,160){\vector(-1,-1){20}}
%\end{picture}
%\end{center}
%\vspace{0.5cm}
%
%Observe also that the family $\{\{2^k an + 2^{k-1} a + b\} :\ k\in\N\}$ witnesses the countable splitting property.
%
%Now, let us consider the case of the Golomb's topology. For any $\{an + b\}\in\B_G$ we can choose such $p\in\mathbb{P}$ that $p\nmid b$ and $p\nmid a+b$. We have: $\{pan + b\} \subseteq \{an + b\}$, $\{pan + a + b\} \subseteq \{an + b\}$, and $\{pan + b\} \cap \{pan + a + b\} = \emptyset$ (if for some $n_1,n_2\in\N$ there would be $pan_1+b=pan_2+a+b$, then $pa(n_1-n_2)=a$, thus $p(n_1-n_2)=1$ -- a contradiction). Moreover, $\{pan + b\}\in\B_G$ as $(a,b)=1 \implies (pa,b)=1$, and $\{pan + a + b\}\in\B_G$ as $(a,b)=1 \implies (a,a+b)=1 \implies (pa,a+b)=1$.
%
%\begin{center}
%\begin{picture}(110,60)
%\put(0,0){\makebox(0,0){$\vdots$}}
%\put(0,10){\makebox(0,0){$\{pan+b\}$}}
%\put(100,10){\makebox(0,0){$\{pan+a+b\}$}}
%\put(50,50){\makebox(0,0){$\{an+b\}$}}
%\put(65,40){\vector(1,-1){20}}
%\put(35,40){\vector(-1,-1){20}}
%\end{picture}
%\end{center}
%\vspace{0.5cm}
%
%The case of the Kirch's topology is a slight modification of the previous construction. Now we additionally require that $p\nmid a$ (hence, as $a\in \SqrFr$, we also have $pa\in \SqrFr$).
%\end{proof}

\begin{prop} \label{remH}
Any base for any Hausdorff topology $\T$ without isolated points has the splitting property.
\end{prop}

\begin{proof}
Let $\B$ be a base for the topology $\T$. Take any $B\in\B$ and two different points $x,y\in B$. As $\T$ is Hausdorff, there exist two open sets $V,W\in\T$ such that $x\in V$, $y\in W$, and $V\cap W = \emptyset$. Let $V':=B\cap V$, $W':=B\cap W$. $V'$ and $W'$ are open, so each of them contains a basic set: $B_{V'}$ and $B_{W'}$, respectively. Now, it is clear that $B_{V'}\cup B_{W'}\subseteq B$ and $B_{V'}\cap B_{W'}=\emptyset$.
\end{proof}

As an immediate consequence we obtain

\begin{cor}
The families $\B_F$, $\B_G$, and $\B_K$ have the splitting property.
\end{cor}

\section{Properties of the three ideals}\label{examples}

At first, we present an example showing that the ideals of Furstenberg, Golomb, and Kirch contain some infinite set.

\begin{ex} 
The set $A = \{n! :\ n\in\N\}$ belongs to $\I_F$, $\I_G$, and $\I_K$.
\end{ex}

%\begin{proof}
%We need to show that for any $\arithseq{a}{b} \in \B_F$ there exists $\arithseq{c}{d} \in \B_F$ with $\arithseq{c}{d} \subseteq \arithseq{a}{b}$, such that $\arithseq{c}{d}\cap A = \emptyset$. Let us fix $\arithseq{a}{b} \in \B_F$. Note that in the set $A$ all but finitely many elements (for $n\geq a$) are divisible by $a$. Consider the two cases:
%\begin{itemize}
% \item[(i)] $b\neq a$. Set $s:= \min \{n :\ an+b>a!\}$. Take $\arithseq{c}{d} := \arithseq{a(s+1)}{as+b}\subseteq \arithseq{a}{b}$. Since $as+b\leq as+a = a(s+1)$, we know that $\arithseq{c}{d}\in \B_F$. Moreover, $as+b>a!$ and all elements of $\arithseq{c}{d}$ are not divisible by $a$, so $\arithseq{c}{d}\cap A = \emptyset$.
% \item[(ii)] $b=a$. Then $\arithseq{a}{b} = \arithseq{a}{a}$, and we can "split" it into two disjoint subsets: $\arithseq{a}{a} = \arithseq{2a}{a}\cup \arithseq{2a}{2a}$. In the set $A$ all but finitely many elements (for $n\geq 2a$) are divisible by $2a$, so almost all elements belong to $\arithseq{2a}{2a}$. Set $s:= \min \{n :\ 2an+a>(2a)!\}$. Take $\arithseq{c}{d} := \arithseq{2a(s+1)}{2as+a}\subseteq \arithseq{a}{a}$. Since $2as+a\leq 2as+2a = 2a(s+1)$, we know that $\arithseq{c}{d}\in \B_F$. Moreover, $2as+a>(2a)!$ and all elements of $\arithseq{c}{d}$ are not divisible by $2a$, so $\arithseq{c}{d}\cap A = \emptyset$.
%\end{itemize}
%
%Note that instead we could have used the fact that the Furstenberg's topology is Hausdorff. The crucial observation is that all but finitely many elements from the set $A$ are divisible by some constant $a$. Since all singleton sets are closed, $\arithseq{a}{b}$ without finitely many points is open and disjoint from $A$ for $b<a$ (case (i)), hence it contains a basic set disjoint from $A$. Case (ii) uses the splitting property, which also follows from $\T_F$ being Hausdorff (see Proposition \ref{remH}).
%
%Thus, the proof for $\I_G$ and $\I_K$ (both $\T_G$ and $\T_K$ are also Hausdorff) goes similarly (in basic sets of these topologies we assume that $b<a$, so we only need to consider case (i)).
%\end{proof}
%%%@@@ alternatywny, znacznie uproszczony dowod zaproponowany przez Recenzenta,
%%%@@@ poprzedni powedrowal na razie w komentarz
\begin{proof}
We show that $A$ is closed with respect to $\T_K$, and thus
is closed with respect to each of the three topologies. To see this, suppose
that $x$ is an accumulation point of $A$ with respect to $\T_K$. 
Pick $p\in\mathbb{P}$ with $p > x$. Then $\arithseq{p}{x}$ is a 
neighborhood of $x$ which hits $A$ in only finitely many points.
  Next note that the interior of $A$ is empty with respect to each 
topology because $A$ contains no three term arithmetic progression.
\end{proof}

The following simple examples show that in some way the Furstenberg's ideal significantly differs from the other two ideals (namely, the Golomb's and the Kirch's ideal).

\begin{ex}[{\cite[Section 5]{Szczuka4}}] \label{primes}
$\mathbb{P}\in \I_F$, but $\mathbb{P}$ is dense in $\T_G$ and $\T_K$ (therefore it does not belong to $\I_G$ nor $\I_K$).
\end{ex}

%%%@@@a moze "the density of" albo "the fact that P is dense"
%%% lub cos w takowym stylu? Anyway, na razie zostawiamy tak:
Notice that the latter fact is an immediate consequence of Dirichlet's Theorem.

\begin{ex} 
$\SqrFr\in \I_F$, but $\SqrFr$ is dense in $\T_G$ and $\T_K$ (therefore it does not belong to $\I_G$ nor $\I_K$).
\end{ex}

\begin{proof}
Firstly, note that the set of squareful numbers is open in $\T_F$ as it is equal to the sum $\bigcup_{p\in\mathbb{P}}{\{p^2 n+p^2\}}$ of arithmetic progressions belonging to $\B_F$. Thus, the set $\SqrFr$ is closed as a complement of an open set (and hence it is equal to its closure). It suffices to show that $\SqrFr$ has empty interior.\\
Let us observe that in every arithmetic progression one can find a squareful number. Indeed, for an arbitrary arithmetic progression $\arithseq{a}{b}$ (with $a,b\in\N$) put $n_0 := a^2 +ab+2a+2b+1$. 
Then,
$$an_0 +b = a(a^2 +ab+2a+2b+1)+b = a((a+1)(a+b+1)+b)+b = $$
%%%@@@ ponizsze dwie linijki wyliczenia zamaskowane ze wzgledu na sugestie Recenzenta
%%%$$= a(a+1)(a+b+1)+(a+1)b = (a+1)(a(a+b+1)+b) =$$
%%%$$= (a+1)(a(a+1)+b(a+1))=$$
$$ (a+1)^2 (a+b),$$
which is always a squareful number, because $a+1  \neq 1$. Hence, no arithmetic progression (and, in particular, no set from $\B_F$) can be contained in $\SqrFr$, which proves that $\SqrFr$ has empty interior, and thus it is nowhere dense in $\T_F$.

The density of $\SqrFr$ in $\T_G$ and $\T_K$ follows from the fact that $\mathbb{P}\subseteq \SqrFr$ and $\mathbb{P}$ is dense in $\T_G$ and $\T_K$ (hence so is every superset of $\mathbb{P}$).
\end{proof}

%%%%%%@@@ przeniesione nad Example 2.4 (czyli ten example ponizej)- poczatek fragmentu
\begin{thm}
$\I_K \subseteq \I_G$.
\end{thm}

\begin{proof}
We will show that $X \not\in \I_G \implies X \not\in \I_K$.\\
Suppose that $X \not\in \I_G$. Then there exists $\arithseq{a}{b}\in \B_G$ such that for every $\arithseq{c}{d}\subseteq \arithseq{a}{b}$ with $\arithseq{c}{d}\in \T_G\setminus\{\emptyset\}$ we have $X\cap \arithseq{c}{d} \neq \emptyset$.
We need to show that there exists $\arithseq{a'}{b'}\in \B_K$ such that for every $\arithseq{c'}{d'}\subseteq \arithseq{a'}{b'}$ with $\arithseq{c'}{d'}\in \B_K$ we have $X\cap \arithseq{c'}{d'} \neq \emptyset$.
Let $a' := \prod_{p\in\Theta(a)}{p}$ (i.e., the "square-free part" of $a$) and $b' := b \mod a'$. Then, $\arithseq{a'}{b'} \in \B_K$ since $a' \in \SqrFr$, $(a',b') = 1$ (as $(a,b) = 1$), and $b'<a'$.
Take any $\arithseq{c'}{d'} \subseteq \arithseq{a'}{b'}$ with $\arithseq{c'}{d'}\in \B_K$ (then $c' \in \SqrFr$, $(c',d')=1$, $d'<c'$, and $a'\mid c'$).
Note that $\arithseq{a}{b}\subseteq \arithseq{a'}{b'}$, $\arithseq{c'}{d'}\subseteq \arithseq{a'}{b'}$, and observe that $\left(\frac{a}{a'}, \frac{c'}{a'}\right) = 1$ -- the only prime factors of $\frac{a}{a'}$ are those that have already appeared in the factorization of $a'$, whereas all prime factors of $\frac{c'}{a'}$ must be different from the prime factors of $a'$ since $c'$ is square-free. 
Hence, $A:= \arithseq{a}{b}\cap\arithseq{c'}{d'} \neq \emptyset$ (by Lemma \ref{lemCRT}). Moreover, $A$ is an arithmetic progression, it belongs to $\T_G$ (as an intersection of two open sets in the Golomb's topology), and, obviously, $A \subseteq \arithseq{a}{b}$. From the assumption we know that $X\cap A \neq \emptyset$. Since we also have that $A \subseteq \arithseq{c'}{d'}$, it is clear that $X\cap \arithseq{c'}{d'} \neq \emptyset$.
\end{proof}
%%%%%%@@@ przeniesione nad Example 2.4 (czyli ten example ponizej)- koniec fragmentu

\begin{ex} \label{even}
The set of even numbers $\arithseq{2}{2}$ is in $\I_K$ hence in $\I_G$, but it belongs to the base for $\T_F$ (therefore $\arithseq{2}{2}\notin \I_F$).
\end{ex}

\begin{proof}
We need to show that for any $\arithseq{a}{b} \in \B_G$ there exists $\arithseq{c}{d} \in \B_G$ with $\arithseq{c}{d} \subseteq \arithseq{a}{b}$, such that $\arithseq{c}{d} \cap \arithseq{2}{2} = \emptyset$. Let us fix $\arithseq{a}{b} \in \B_G$ (we assume that $(a,b)=1$ and $b<a$) and consider the two cases:
\begin{itemize}
 \item[(i)] $2\mid b$. Then $2 \nmid a$ (otherwise, $a$ and $b$ would not be coprime). Take $\arithseq{c}{d} := \arithseq{2a}{a+b}$. As $(a,b)=1$, we know that $(a,a+b)=1$, and as $2 \nmid a$, we have $(2a,a+b)=1$. Thus, $\arithseq{2a}{a+b}\in\B_G$ and $\arithseq{2a}{a+b}\subseteq\arithseq{a}{b}$. Moreover, $\arithseq{2a}{a+b}$ consists only of odd numbers, so $\arithseq{2a}{a+b}\cap \arithseq{2}{2} = \emptyset$.
 \item[(ii)] $2\nmid b$. Take $\arithseq{c}{d} := \arithseq{2a}{b}$. As $2\nmid b$, we have $(2a,b)=1$. Thus, $\arithseq{2a}{b}\in\B_G$ and $\arithseq{2a}{b}\subseteq\arithseq{a}{b}$. Moreover, $\arithseq{2a}{b}$ consists only of odd numbers, so $\arithseq{2a}{b}\cap \arithseq{2}{2} = \emptyset$.
\end{itemize}

The proof for $\I_K$ goes similarly. In item (i) we additionally need to observe that if $2 \nmid a$ and $a$ is square-free, then $2a$ will also be square-free. In item (ii), if $2 \nmid a$, we use the same argument as above, and if $2\mid a$, we take $\arithseq{c}{d} := \arithseq{a}{b}$ as it is already disjoint from the set of even numbers $\arithseq{2}{2}$.
\end{proof}

The proof of the previous result can easily be generalized for sets of multiples of any prime number, as follows:

\begin{ex} 
For any $p\in\mathbb{P}$ we have $\arithseq{p}{p}\in \I_K \setminus \I_F$.
\end{ex}

\begin{cor}
If $p_1, \ldots, p_k \in \mathbb{P}$, then $\{n\in\N :\ p_1\mid n\ \vee \ldots \vee\ p_k\mid n\}\in \I_K \setminus \I_F$.
\end{cor}

Now, let us investigate a relation between the ideals of Golomb and Kirch.

While searching for an example distinguishing these ideals, at first we have obtained a partial result, suggesting that an example of a set witnessing the lack of inclusion between them cannot be found among the sets from $\B_G \setminus \B_K$:

%%%@@@ Recenzent napisal tak: "There doesn't seem to be any motive for including Proposition
%%% 2.7 (czyli to ponizej). Pytanie: Czy usuwamy czy tez wymyslamy jakis "motive"
%%% dla uzasadnienia racji bytu tego fragmentu?
\begin{prop}
Every set of the form $\arithseq{2s}{q}$, where $s \in\SqrFr$, $2\mid s$, $(s,q)=1$, and $q<s$, is in $\B_G$ but not in $\B_K$, but it does not belong to $\I_K$.
\end{prop}

\begin{proof}
Let us first observe that $\arithseq{2s}{q}$, as defined above, satisfies $(2s,q)=1$ and $q<2s$ -- therefore it is in $\B_G$. Moreover, $2s$ is not square-free, so it cannot be in $\B_K$.
Now, we want to show that there exists $\arithseq{a}{b}\in \B_K$ such that for every $\arithseq{c}{d}\subseteq \arithseq{a}{b}$ with $\arithseq{c}{d}\in \B_K$ we have $\arithseq{c}{d}\cap \arithseq{2s}{q} \neq \emptyset$. Let $\arithseq{a}{b} := \arithseq{s}{q}$ (it belongs to $\B_K$ since $s\in\SqrFr$, $(s,q)=1$, and $q<s$). Take any $\arithseq{c}{d}\subseteq \arithseq{s}{q}$ with $\arithseq{c}{d}\in \B_K$ -- we then know that $c\in\SqrFr$. Suppose that $\arithseq{c}{d}\cap \arithseq{2s}{q} = \emptyset$. As $\arithseq{c}{d}\subseteq \arithseq{s}{q}$ and $\arithseq{s}{q} \setminus \arithseq{2s}{q} = \arithseq{2s}{q+s}$, it means that $\arithseq{c}{d}\subseteq \arithseq{2s}{q+s}$ -- but then $2s$ must divide $c$, which hence cannot be square-free. A contradiction ends the proof.
\end{proof}

Eventually, we managed to prove an inclusion between the ideals of Golomb and Kirch -- however, it turns out that they are not the same. These results will be shown in the next two theorems. In their proofs we will use a technical lemma:
 
\begin{lem} \label{lemCRT}
Assume that $a,b,a_1,b_1,a_2,b_2 \in\N$. If $\arithseq{a_1}{b_1}\subseteq \arithseq{a}{b}$, $\arithseq{a_2}{b_2}\subseteq \arithseq{a}{b}$, and $\left(\frac{a_1}{a},\frac{a_2}{a}\right)=1$, then the intersection $\arithseq{a_1}{b_1}\cap\arithseq{a_2}{b_2}$ is nonempty (and hence it is an arithmetic progression). 
\end{lem}

\begin{proof}
Let $f\colon\N\to \arithseq{a}{b}$ be a bijection such that $f(n) = an+b$. Then: 
$$f^{-1}[\{a_1 n+b_1\}] = \left\{\frac{a_1}{a} n+\frac{b_1-b}{a}\right\},\ \ \ f^{-1}[\{a_2 n+b_2\}] = \left\{\frac{a_2}{a} n+\frac{b_2-b}{a}\right\},$$
and since $\left(\frac{a_1}{a},\frac{a_2}{a}\right)=1$, the Chinese remainder theorem guarantees that their intersection is nonempty. Thus,
$$f^{-1}[\arithseq{a_1}{b_1}\cap\arithseq{a_2}{b_2}] = f^{-1}[\arithseq{a_1}{b_1}]\cap f^{-1}[\arithseq{a_2}{b_2}]\neq\emptyset,$$
and hence
$$\arithseq{a_1}{b_1}\cap\arithseq{a_2}{b_2} = f[f^{-1}[\{a_1 n+b_1\}\cap\{a_2 n+b_2\}]] \neq\emptyset.$$
\end{proof}

\begin{thm}
%%%$\I_G \not\subseteq \I_K$.
%%%@@@ zamieniam wedle sugestii Recenzenta na:
$(\I_G\cap \I_F) \not\subseteq \I_K$.
\end{thm}

\begin{proof}
Define $\mathcal{C} := \{\arithseq{a}{b}\in \B_K :\ \arithseq{a}{b}\subseteq \arithseq{2}{1}\}$. Let $\{C_k :\ k\in\N\}$ be an enumeration of $\mathcal{C}$.
%%%We will construct a set $X \in \I_G \setminus \I_K$. 
%%%@@@ zamieniam wedle sugestii Recenzenta na:
We will construct a set $X \in (\I_G\cap \I_F) \setminus \I_K$. 
For every $k\in\N$ pick $x_k$ such that:
\begin{itemize}
	\item $x_k\in C_k$,
	\item $x_k\in \arithseq{2^k}{1}$.
\end{itemize}
Observe that such construction is possible since $C_k \cap \arithseq{2^k}{1}$ is always nonempty (if $C_k = \arithseq{a_k}{b_k}$, then $a_k$ is even and square-free, and $\left(\frac{a_k}{2},\frac{2^k}{2}\right)=\left(\frac{a_k}{2},2^{k-1}\right)=1$ as $\frac{a_k}{2}$ is odd; both $\arithseq{a_k}{b_k}$ and $\{2^k n+1\}$ are subsequences of $\arithseq{2}{1}$, so, by Lemma \ref{lemCRT}, their intersection is nonempty). Let $X := \{x_k :\ k\in\N\}$.
Firstly, note that $X \notin \I_K$. Indeed, the set $\arithseq{2}{1}\in\B_K$ has a property that for every $\arithseq{c}{d}\subseteq \arithseq{2}{1}$ with $\arithseq{c}{d}\in \B_K$ (so $\arithseq{c}{d}=C_{k_0}$ for some $k_0\in\N$) we have $X\cap \arithseq{c}{d} \neq \emptyset$ (as it contains $x_{k_0}$).
%%%@@@ teraz nastepuje rozumowanie zasugerowane przez Recenzenta,
%%%@@@ TODO: nalezy koniecznie przesledzic poprawosc rozumowania,
%%% i jesli bedzie OK, to trzeba bedzie zapewne usunac stare rozumowanie ponizej
%%% konca wkopiowanego tekstu:
  Take any $\arithseq{a}{b} \in \B_F$. We will show that there exists
nonempty $V \subseteq \arithseq{a}{b}$ with $X\cap V = \emptyset$ such
that $V\in \T_F$ and, if $\arithseq{a}{b}\in \B_G$, then $V\in\T_G$.

  Assume first that $2\not | a$. Let 
$V = (\arithseq{a}{b} \cap \arithseq{4}{3}) \setminus \{x_1\}$. 
Then $V \not= \emptyset$ because $(a,4) = 1$ so 
$V\in \T_F$ and, if $\arithseq{a}{b}\in \B_G$, then $V\in \T_G$.
If $k > 1$, then $x_k \in \arithseq{4}{1}$ so $V\cap X = \emptyset$.
  
  Now assume that $2|a$ and pick the largest $m\in \N$
such that $2^m|a$. If $b$ is even then $\arithseq{a}{b}\cap X = \emptyset$
and we may let $V = \arithseq{a}{b}$. Thus we may assume
that $b$ is odd. If $b\not\equiv 1 (\mod 2^m)$, let
$V = \arithseq{a}{b}\setminus \{x_1,\ldots, x_{m-1}\}$.
Then $\arithseq{a}{b}\cap \arithseq{2^m}{1} = \emptyset$ so $V\cap X = \emptyset$.

  Thus we may assume that $b\equiv 1 (\mod 2^m)$. Let 
\[
V = (\arithseq{a}{b} \cap \arithseq{2^{m+1}}{1 + 2^m}) \setminus \{x_1,\ldots, x_{m-1}\}.
\]
Then $V \not= \emptyset$ because $(\frac{a}{2^m} : \frac{2^{m+1}}{2^m}) = 1$
and $\arithseq{a}{b}$ and $\arithseq{2^{m+1}}{1 + 2^m}$
are contained in $\arithseq{2^m}{1}$. 
If $k > m$, then $x_k \in \arithseq{2^m}{1}$ so 
$V\cap X = \emptyset$ and, if $\arithseq{a}{b} \in \B_G$, then
$V \in \T_G$.

%%% koniec wkopiowanego tekstu, zatem prawdopodobnie mozna ponizszy
%%% nasz fragment usunac?
%Now, we will prove that $X \in \I_G$. Take any $\arithseq{a}{b}\in\B_G$. We want to show that there exists $V\subseteq \arithseq{a}{b}$ with $V\in \T_G\setminus\{\emptyset\}$, such that $X\cap V = \emptyset$. \\
%Assume that $2\nmid a$. Let $V := (\arithseq{a}{b} \cap \arithseq{4}{3})\setminus\{x_1\}$. $V$ is open in $\T_G$ (as a difference of an open set and a closed set) and nonempty (because $(a,4)=1$). Moreover, for every $k\geq 2$ we have that $x_k\in \arithseq{4}{1}$, which is disjoint from $\arithseq{4}{3}$, and hence $X\cap V = \emptyset$.
%Now, consider the following cases:
%\begin{itemize}
% \item $2\mid a$ and $4\nmid a$. Observe that then $2 \nmid b$ (otherwise, $a$ and $b$ would not be coprime), so $\arithseq{a}{b}\subseteq \arithseq{2}{1}$. Again, let $V := (\arithseq{a}{b} \cap \arithseq{4}{3})\setminus\{x_1\}$. $V$ is open in $\T_G$ and nonempty (because $\left(\frac{a}{2},\frac{4}{2}\right)=\left(\frac{a}{2},2\right)=1$ as $\frac{a}{2}$ is odd, and both $\{an+b\}$ and $\arithseq{4}{3}$ are subsequences of $\arithseq{2}{1}$). Moreover, as above, $X\cap V = \emptyset$.
% \item $4\mid a$ and $8\nmid a$. Then, as in the previous case, $\arithseq{a}{b}\subseteq \arithseq{2}{1}$. Now, either $b\equiv 1\ (\modulo 4)$ or $b\equiv 3\ (\modulo 4)$. Thus, either $\arithseq{a}{b}\subseteq \arithseq{4}{1}$ or $\arithseq{a}{b}\subseteq \arithseq{4}{3}$. In the first case, let $V := (\{an+b\} \cap \{8n+5\})\setminus\{x_1, x_2\}$. $V$ is open in $\T_G$ and nonempty (because $\left(\frac{a}{4},\frac{8}{4}\right)=\left(\frac{a}{4},2\right)=1$ as $\frac{a}{4}$ is odd, and both $\{an+b\}$ and $\{8n+5\}$ are subsequences of $\{4n+1\}$). Moreover, for every $k\geq 3$ we have that $x_k\in \{8n+1\}$, which is disjoint from $\{8n+5\}$, and hence $X\cap V = \emptyset$. In the second case, take $V := \{an+b\} \setminus\{x_1\}$. $V$ is open in $\T_G$ and nonempty. Moreover, for every $k\geq 2$ we have that $x_k\in \{4n+1\}$, which is disjoint from $\{4n+3\}$, and hence $X\cap V = \emptyset$.
%\end{itemize}
%Generally, for $m\in\N$:
%\begin{itemize}
% \item $2^m\mid a$ and $2^{m+1}\nmid a$. Observe that then $2 \nmid b$ (otherwise, $a$ and $b$ would not be coprime), so $\arithseq{a}{b}\subseteq \arithseq{2}{1}$. Now, either $b\equiv 1\ (\modulo 2^m)$ or $b \not\equiv 1\ (\modulo 2^m)$. In the first case, let $V := (\arithseq{a}{b} \cap \{2^{m+1}n+2^m+1\})\setminus\{x_1, x_2,\ldots, x_m\}$. $V$ is open in $\T_G$ and nonempty (because $\left(\frac{a}{2^m},\frac{2^{m+1}}{2^m}\right)=\left(\frac{a}{2^m},2\right)=1$ as $\frac{a}{2^m}$ is odd, and both $\arithseq{a}{b}$ and $\arithseq{2^{m+1}}{2^m+1}$ are subsequences of $\{2^m n+1\}$). Moreover, for every $k\geq m+1$ we have that $x_k\in \{2^{m+1}n+1\}$, which is disjoint from $\{2^{m+1}n+2^m+1\}$, and hence $X\cap V = \emptyset$. In the second case, take $V := \{an+b\} \setminus\{x_1, x_2,\ldots, x_{m-1}\}$. $V$ is open in $\T_G$ and nonempty. Moreover, for every $k\geq m$ we have that $x_k\in \arithseq{2^m}{1}$, which is disjoint from $\arithseq{2^m}{b}$, and hence $X\cap V = \emptyset$.
%\end{itemize}
\end{proof}


\section{Marczewski-Burstin representations}

Recall that a set $A$ is \emph{nowhere dense in topology $\T$} if:
$$\forall_{U\in\T\setminus\{\emptyset\}}\ \exists_{V\in\T\setminus\{\emptyset\},\ V\subseteq U}\ A\cap V = \emptyset.$$
The authors of \cite{MB} observed that the scheme defining the family of nowhere dense sets turns out to be interesting also if the family of open sets is substituted by a family of nonempty subsets of an arbitrary set $X$.

\begin{df}[\cite{MB}] 
For an $X\neq\emptyset$ and a given family $\mathcal{F}\subseteq \mathcal{P}(X)\setminus\{\emptyset\}$,
$$\MB(\mathcal{F}) := \left\{A\subseteq X :\ \forall_{F\in\mathcal{F}}\ \exists_{G\in\mathcal{F},\ G\subseteq F}\ A\cap G=\emptyset\right\}$$
is an ideal, called the \emph{Marczewski ideal}.
\end{df}

Clearly, always $\mathcal{F}\cap \MB(\mathcal{F})=\emptyset$.

In the 2000s, there has appeared a series of articles (e.g., \cite{MB}, \cite{MB2}, \cite{MB3}, \cite{MB4}), where the authors study these ideals (and corresponding algebras) with respect to different generating families.

Note that in the case when $\mathcal{F}$ consists of all perfect subsets of a given Polish space, $\MB(\mathcal{F})$ is the family of classical Marczewski $(s^0)$-sets (see \cite{Sz}), and, obviously, if $(X,\T)$ is a given topological space, $\MB(\T\setminus\{\emptyset\})$ is the family of all nowhere dense sets in this topology.\\


If a given ideal $\I$ on a set $X$ can be represented as $\MB(\mathcal{F})$ for some family $\mathcal{F}\subseteq \mathcal{P}(X)\setminus\{\emptyset\}$, we say that it is \emph{Marczewski-Burstin representable by $\mathcal{F}$}. 

\begin{df}
Let us call an ideal $\I\subseteq \mathcal{P}(\N)$ \emph{Marczewski-Burstin countably representable} (briefly: $\MBC$) if there exists a countable family $\mathcal{F}\subseteq \InfSubs$ such that $\I = \MB(\mathcal{F})$.
\end{df}

\begin{remark} 
$\Fin = \MB(\mathcal{F})$, where $\mathcal{F}= \left\{[n, +\infty)\cap\N :\ n\in\N\right\}$. Hence, $\Fin$ is $\MBC$.
\end{remark}

\begin{remark} 
Obviously, the ideals $\I_F$, $\I_G$, and $\I_K$ are $\MBC$.
\end{remark}

Let us now investigate some properties of the $\MBC$ ideals.

We say that an ideal $\mathcal{I}\subseteq\mathcal{P}(\N)$ is \emph{tall} if any infinite set in $\N$ contains an infinite subset that belongs to $\mathcal{I}$.

\begin{df}
Let us say that a family $\mathcal{F}\subseteq \InfSubs$ has the \emph{base-like property} if:
$$\forall_{F_1, F_2\in\mathcal{F},\ F_1\cap F_2\neq\emptyset}\ \exists_{H\in\mathcal{F}}\ H\subseteq F_1\cap F_2.$$
\end{df}

Recall that a family $\mathcal{F} \subseteq \InfSubs$ has the \emph{splitting property} if: 
$$\forall_{F\in\mathcal{F}}\ \exists_{F_1,F_2\in\mathcal{F},\ F_1\cup F_2 \subseteq F}\ F_1\cap F_2 = \emptyset.$$

\begin{thm} \label{thmtall}
Let $\mathcal{F}\subseteq \InfSubs$ be a countable family. Suppose that: 
\begin{itemize}
	\item[$(i)$] $\mathcal{F}$ has the base-like property,
	\item[$(ii)$] $\mathcal{F}$ has the splitting property.
\end{itemize}
Then the ideal $\I=\MB(\mathcal{F})$ is tall.
\end{thm}

\begin{proof}
Let $A \subseteq\N$ be any infinite set. 
The case that $A\in \MB(\mathcal{F})$ is trivial, 
so we assume that $A\notin \MB(\mathcal{F})$. Then there exists a set $F\in\mathcal{F}$ such that for every $G\in\mathcal{F}$ with $G\subseteq F$ we have $A\cap G \neq\emptyset$. By the splitting property, find a family $\{F_i\}_{i\in\N} \subseteq\mathcal{F}$ such that:
\begin{itemize}
\item $\forall_{i}\ F_i \subseteq F$,
\item $\forall_{i\neq j}\ F_i\cap F_j =\emptyset$.
\end{itemize}
Since $\forall_{i}\ A\cap F_i \neq\emptyset$, for each $i$ let us choose any element $b_i \in A\cap F_i$, and define $B:=\{b_i :\ i\in\N\}$. We will show that $B\in\MB(\mathcal{F})$ -- this will prove that $\I$ is tall, as $B$ is an infinite set contained in $A$.\\
Fix any $G\in \mathcal{F}$. If $B\cap G=\emptyset$, then the proof is finished. If $B\cap G \neq\emptyset$, let $i_0\in\N$ be such that $b_{i_0}\in G$. Then $F_{i_0}\cap G \neq\emptyset$, and, by the base-like property, there exists $H\in\mathcal{F}$ such that $H\subseteq F_{i_0}\cap G$. Therefore, $|B\cap H|\leq 1$ (because $B\cap H\subseteq\{b_{i_0}\}$). Now, using the splitting property for $H$, we obtain that there exists $H^*\in\mathcal{F}$ with $H^*\subseteq H$, such that $b_{i_0}\notin H^*$, so $B\cap H^* =\emptyset$. Hence, $B\in\MB(\mathcal{F})$.
\end{proof}

\begin{remark}\label{remtall+}
Let $\mathcal{F}$ be a family meeting the assumptions of the previous theorem.
Notice that a similar proof shows that if $\{F_i\}_{i\in\N}$ is a family of pairwise disjoint elements from $\mathcal{F}$ and $A\subseteq \N$ is such that:
\begin{itemize}
	\item $A\subseteq \bigcup_{i\in\N}{F_i}$,
	\item $\forall_{i\in\N}\ A\cap F_i\in\Fin$,
\end{itemize}
then $A\in\MB(\mathcal{F})$.
\end{remark}

\begin{proof}
Take any $A\subseteq\N$ fulfilling the above conditions, and fix $F\in\mathcal{F}$. If $A\cap F=\emptyset$, then the proof is finished. Now, let us assume that $A\cap F \neq\emptyset$. Choose $i\in\N$ such that $F_i\cap F \neq\emptyset$. By the base-like property, there exists $H\in\mathcal{F}$ such that $H\subseteq F_i\cap F$. Therefore, $A\cap H\in\Fin$ (because $A\cap H\subseteq A\cap F_i$). Again, by "splitting" $H$ sufficiently many times, we obtain that there exists $H^*\in\mathcal{F}$ with $H^*\subseteq H$, such that $A\cap H^* =\emptyset$. Hence, $A\in\MB(\mathcal{F})$.
\end{proof}

\begin{cor} \label{tall}
The ideals $\I_F$, $\I_G$, and $\I_K$ are tall.
\end{cor}

%%%@@@ ta definicja dodana przez sugestie Recenzenta, oczywiscie do wygladzenia,
%%%@@@ choc pewnie moze takowa pozostac?
Recall that an ideal $\I$ on $\N$ is $F_{\sigma\delta}$ if
$\I \subseteq P(\N) \simeq 2^\omega$
is an $F_{\sigma\delta}$ in the usual product topology
of $2^\omega$.

Now, we will show that $\I_F$, $\I_G$, and $\I_K$ are $F_{\sigma\delta}$ but not $F_{\sigma}$ ideals.

In fact, let us formulate slightly more general results:

\begin{remark}
Every $\MBC$ ideal is of type $F_{\sigma\delta}$.
\end{remark}

\begin{proof}
Suppose that there exists a countable family $\mathcal{F}\subseteq \InfSubs$ such that $\I = \MB(\mathcal{F})$. Then,
$$X\in \MB(\mathcal{F})\ \Longleftrightarrow\ X\in \bigcap_{F\in\mathcal{F}}\ \bigcup_{G\in\mathcal{F},\ G\subseteq F}\ \{A\subseteq\N :\ A\cap G=\emptyset\},$$
which is a set of type $F_{\sigma\delta}$ since the sets $\{A\subseteq\N :\ A\cap G=\emptyset\}$ are closed in $\mathcal{P}(\N)$.
\end{proof}

\begin{thm} \label{thmFsigma}
Let $\mathcal{F}\subseteq \InfSubs$ be a countable family. Suppose that: 
\begin{itemize}
	\item[$(i)$] $\mathcal{F}$ has the base-like property,
	\item[$(ii)$] $\mathcal{F}$ has the splitting property.
\end{itemize} 
Then the ideal $\I=\MB(\mathcal{F})$ is not of type $F_{\sigma}$.
\end{thm}

\begin{proof}
Suppose that $\I=\MB(\mathcal{F})$ is an $F_\sigma$ ideal.
By Mazur's characterization from \cite{Maz}, there exists a lower semicontinuous submeasure $\phi\colon\mathcal{P}(\N)\to [0, \infty]$, i.e., a function such that for any $A,B\subseteq\N$:
\begin{itemize}
\item $\phi(\emptyset)=0$ and $\phi(\N)>0$,
\item $\phi(\{n\})<\infty$ for every $n\in\N$,
\item $A\subseteq B \implies\phi(A)\leq \phi(B)$,
\item $\phi(A\cup B) \leq \phi(A) + \phi(B)$,
\item \label{continuity-condition} $\phi(A)=\lim_{n\to\infty} \phi(A \cap \{1,\ldots,n\})$ (lower semicontinuity),
\end{itemize}
for which $\I= \Fin(\phi)=\{A\subseteq \N :\ \phi(A)<\infty\}$.\\
Using the splitting property, fix a family $\{F_i\}_{i\in\N}$ of pairwise disjoint elements from $\mathcal{F}$.
Obviously, $\forall_{i}\ F_i\notin\I$, so $\phi(F_i)=\infty$. For each $i\in\N$ let us choose (by the lower semicontinuity of $\phi$) a finite set $A_i\subseteq F_i$ such that $\phi(A_i)>i$. Define $A:=\bigcup_{i\in\N}{A_i}$. From Remark \ref{remtall+} it follows that $A\in S^0(\mathcal{F}) = \I$. On the other hand, $\forall_{i}\ A_i \subseteq A$, so $i<\phi(A_i)\leq \phi(A)$. Therefore, $\phi(A)=\infty$ and hence $A\notin\I$, which is a contradiction.
\end{proof}

\begin{cor}
$\I_F$, $\I_G$, and $\I_K$ are $F_{\sigma\delta}$ but not $F_{\sigma}$ ideals.
\end{cor}


\section{Topological representations}

We show that our new ideals (on a countable set) may be connected to some $\sigma$-ideals in separable metrizable spaces. This connection has been introduced by M. Sabok and J. Zapletal in \cite{Sabok}. 

\begin{df}[\cite{Sabok}]
Suppose that $X$ is a separable metrizable space, $D\subseteq X$ -- a dense countable set, and $I$ -- a $\sigma$-ideal on $X$, containing all singletons. Then,
$$\mathcal{J}_I:=\left\{A\subseteq D :\ \cl(A)\in I\right\},$$
where $\cl$ denotes the closure operation in $X$,
is an ideal on $D$. Given an ideal $\mathcal{I}$ on $\N$, we say that $\mathcal{I}$ \emph{has a topological representation} if there are $I,D,X$ as above, for which $\mathcal{I}$ is isomorphic to $\mathcal{J}_I$ (i.e., there exists a bijection $f\colon \N\to D$ such that $A\in\I \Leftrightarrow f[A]\in\mathcal{J}_I$). In such a case we say that $\mathcal{I}$ \emph{is represented on $X$ by $I$}.
\end{df}

If $\mathcal{I}\subseteq\mathcal{P}(\N)$ has a topological representation, then it can be represented on the Cantor space $2^\N$ by a $\sigma$-ideal generated by a family of compact nowhere dense sets (see \cite[Corollary 1.3]{Adas}). 

\begin{df} Let $\mathcal{I}$ be an ideal on $\N$.
\begin{enumerate}
\item[(i)] $\mathcal{I}$ is \emph{$\omega$-$+$-diagonalizable} if there is a countable family $\{X_n\}_{n\in\N}$ of subsets of $\N$, such that for any $A\in \mathcal{I}$ there is $n\in\N$ with $A\cap X_n=\emptyset$.
\item[(ii)] $\mathcal{I}$ is \emph{countably separated} if there is a countable family $\{X_n\}_{n\in\N}$ of subsets of $\N$, such that for any $A\in \mathcal{I}$ and $B\notin \mathcal{I}$ there is $n\in\N$ with $A\cap X_n=\emptyset$ and $B\cap X_n\notin \mathcal{I}$. % In such a case we say that the family $\{X_n:\ n\in\N\}$ \emph{separates} $\mathcal{I}$.
\item[(iii)] $\mathcal{I}$ is \emph{weakly selective} if for every partition $\{X_n\}_{n\in\N}$ of $\N$, such that $X_i\in\I$ for $i\geq 2$ and $\bigcup_{i\geq 2}{X_i} \notin\I$, there exists a selector of the partition $\{X_n\}_{n\in\N}$, which does not belong to $\I$.
\end{enumerate}
\end{df}

We obtain an immediate remark:

\begin{remark}
The ideals $\I_F$, $\I_G$, and $\I_K$ are $\omega$-$+$-diagonalizable.
\end{remark}

\begin{proof}
Let the base $\B_F$ for the Furstenberg's topology be the required countable family $\{X_n\}_{n\in\N}$ of subsets of $\N$. Take any $A\in \I_F$. We know that for every $X_m \in \B_F$ there exists $X_k \in \B_F$ with $X_k \subseteq X_m$, such that $A\cap X_k = \emptyset$. Thus, $X_k$ satisfies the condition from the definition of $\omega$-$+$-diagonalizability.

The proof for $\I_G$ and $\I_K$ goes analogously.
\end{proof}

It can also be easily proved that $\I_F$, $\I_G$, and $\I_K$ are countably separated -- it suffices to take the bases for the respective topologies as the separating families $\{X_n\}_{n\in\N}$.

In fact, the above reasoning can be conducted for every Marczewski-Burstin countably representable ideal (the form of the basic sets does not play any important role here). Let us then formulate the following, more general, theorem:

\begin{thm}
Let $\I$ be an ideal on $\N$.
\begin{itemize}
\item[$(i)$] If $\I$ is $\MBC$, then $\I$ is countably separated.
\item[$(ii)$] If $\I$ is countably separated, then there exists an $\MBC$ ideal $\J$ such that $\I \subseteq \J$.
\end{itemize}
\end{thm}

\begin{proof}
$(i)$ Suppose that $\I = \MB(\calF)$ for some countable family $\calF \subseteq \InfSubs$. Let us define $\{X_n\}_{n\in\N} := \calF$. Take any $A\in\I$ and $B\notin\I$. We then know that there exists $F\in\calF$ such that for every $G\in\calF$ with $G\subseteq F$ we have $B\cap G \neq\emptyset$. Moreover, notice that $B\cap G \notin\I$. Indeed -- otherwise, we would have that $B\cap G \in\I$, so for some $H\in\calF$ with $H\subseteq G$ there would be $(B\cap G)\cap H = B\cap H =\emptyset$, which is impossible (since, on the other hand, $B\cap H \neq\emptyset$ as $H\subseteq F$). Since $A\in\I$, there exists $X_n \in \calF$ with $X_n \subseteq F$, such that $A\cap X_n = \emptyset$. What is more, $B\cap X_n\notin\I$, and this finishes the proof.

$(ii)$ Suppose that $\{X_n\}_{n\in\N}$ is such that for any $A\in\I$ and $B\notin\I$ there is $n\in\N$ with $A\cap X_n=\emptyset$ and $B\cap X_n\notin\I$. Define
$$\calF := \left\{\bigcap_{i\in S}{X_{i}} :\ S\in\Fin,\ \bigcap_{i\in S}{X_{i}}\notin\I\right\}.$$
Note that $\calF$ is nonempty (without loss of generality, in the family $\{X_n\}_{n\in\N}$ we can take only the sets not belonging to $\I$, so for every $n$ we have $X_n\in\calF$) and countable.
Let $\J := \MB(\calF)$. Suppose that $A\in\I$. In order to check that $\I \subseteq \J$, we need to show that for every $F\in\calF$ there exists $G\in\calF$ with $G\subseteq F$, such that $A\cap G =\emptyset$. Set any $F\in\calF$. Then $F = \bigcap_{i\in S}{X_{i}}$ for some $S\in\Fin$.
Since $F\notin\I$ and $\I$ is countably separated, we can find $X_n$ such that $A\cap X_n=\emptyset$ and $F\cap X_n\notin\I$. Then, $G := \bigcap_{i\in S}{X_{i}} \cap X_n = F\cap X_n \notin\I$, so $G\in\calF$. Moreover, $G\subseteq F$ and $A \cap G = A \cap (F\cap X_n) =\emptyset$. This finishes the proof.
\end{proof}

\begin{cor} \label{cs}
The ideals $\I_F$, $\I_G$, and $\I_K$ are countably separated.
\end{cor}

A. Kwela and P. Zakrzewski showed in \cite[Proposition 4.3]{KwelaZak} that every countably separated ideal on $\N$ is weakly selective -- so we get an immediate corollary:

\begin{cor}
The ideals $\I_F$, $\I_G$, and $\I_K$ are weakly selective.
\end{cor}


In \cite{Adas}, the authors presented the following characterization: 

\begin{thm}[{\cite[Theorem 1.1]{Adas}}]
An ideal on a countable set has a topological representation if and only if it is tall and countably separated.
\end{thm}

From Corollary \ref{tall} and Corollary \ref{cs} it follows that:

\begin{cor}
The ideals $\I_F$, $\I_G$, and $\I_K$ have a topological representation.
\end{cor}

The main motivation for investigating this property was a paper of M. Sabok and J. Zapletal (\cite{Sabok}). The authors have proposed the following conjecture:

\begin{conj}[\cite{Sabok}]
An ideal on a countable set has a topological representation if and only if it is tall, $F_{\sigma\delta}$, and weakly selective. 
\end{conj}

They also prove the part "only if", so the remaining question concerns solely the part "if".
Notice that our ideals fulfill all of the above conditions, and we have shown that they have a topological representation. Hence, we only provide an example confirming the conjecture -- but it does not prove or disprove anything...\\


So far, only two examples of ideals with topological representations have been studied, namely:
%%%@@@ wedlug sugestii Recenzenta dodalismy cl_\R ale moze taki zabieg
%%%@@@ trzeba i wykonac nizej ? Bowiem i tam sa tez definicje w podobnym stylu...
$$\NWD(\Q):=\left\{A\subseteq\mathbb{Q}\cap [0,1] :\ \cl(A) \textrm{ is nowhere dense}\right\},$$
$$\NULL(\Q):=\left\{A\subseteq\mathbb{Q}\cap [0,1] :\ \cl_\R(A) \textrm{ is of Lebesgue measure zero}\right\},$$
which are the ideals represented on $[0,1]$ by the $\sigma$-ideals of meager sets and sets of Lebesgue measure zero, respectively. In \cite{FS}, I. Farah and S. Solecki proved that $\NWD(\Q)$ and $\NULL(\Q)$ are non-isomorphic.

\begin{prop}
$\I_F$ and $\NWD(\Q)$ are isomorphic.
\end{prop}

\begin{proof}
Firstly, note that $\NWD(\Q)$ can be also defined as:
$$\NWD(\Q) := \left\{A\subseteq\mathbb{Q}\cap [0,1] :\ A \textrm{ is nowhere dense}\right\}.$$
Clearly, $\N$ with the Furstenberg's topology is homeomorphic to $\Q$ (with the natural topology), due to a 1920 theorem of Sierpi\'nski, which characterizes the rationals as the unique countable metrizable space without isolated points. Obviously, $\Q$ is homeomorphic to $\Q\cap (0,1)$. Therefore, $\I_F$ is isomorphic to the ideal $\I := \left\{A\subseteq\Q\cap (0,1) :\ A \textrm{ is nowhere dense}\right\}$. It is not hard to observe that $\I$ is isomorphic to $\NWD(\Q)$. Indeed, set any infinite $A\in\I$ and define a function $f \colon \Q\cap (0,1) \to \Q\cap [0,1]$ such that $f\upharpoonright A$ is an arbitrary bijection between $A$ and $A\cup\{0,1\}$, and $f\upharpoonright A^C$ is an identity map on $A^C$. Then $f$ is an isomorphism between the ideals $\I$ and $\NWD(\Q)$.
\end{proof}

Let us now consider the following ideals on $\N$:
$$\NWD :=\left\{A\subseteq\N :\ \cl(b[A]) \textrm{ is nowhere dense}\right\},$$
$$\NULL :=\left\{A\subseteq\N :\ \cl(b[A]) \textrm{ is of Lebesgue measure zero}\right\},$$
where $b\colon\N\to\Q$ is any fixed bijection.

\begin{remark}
Clearly, $\NWD$ is $\MBC$ since it is isomorphic to $\I_F$. In fact, it can be shown that $\NWD = \MB(\mathcal{F})$, where:
$$\mathcal{F} := \{b^{-1}[(p, q)\cap\Q] :\ p < q,\ p, q \in \Q\}.$$
\end{remark}

\begin{thm}
$\INULL := \{A \subseteq\R :\ \cl(A) \textrm{ is of Lebesgue measure zero}\}$ is $\MBC$.
\end{thm}

\begin{proof}
Let $\Seg$ be the family of all finite unions of closed intervals with rational endpoints, which have measure greater than $1$, i.e.:
$$\Seg := \left\{\bigcup_{i=1}^{k}{[p_i,q_i]} :\ k\in\N,\ \forall_{i=1,\ldots,k}\ p_i<q_i,\ p_i,q_i\in\Q,\ \mu\left(\bigcup_{i=1}^{k}{[p_i,q_i]}\right)>1\right\}.$$
We have: $\INULL = \MB(\Seg)$.\\
Indeed, if $A \in \INULL$ and $S\in\Seg$, then $\interior(S) \setminus \cl(A)$ is an open set of measure greater than $1$. Therefore, there exists $S_1\in \Seg$ such that $S_1 \subseteq S \setminus \cl(A)$, and thus $A \cap S_1 =\emptyset$, so $A\in\MB(\Seg)$.\\
On the other hand, suppose that $A\subseteq \R$ is such that $\cl(A)$ has positive measure. By the Lebesgue's density theorem, find $x_0\in\cl(A)$ and $0 < \eta < \frac{1}{2}$ such that $\mu((x_0-\eta, x_0+\eta) \cap \cl(A)) > \eta$. Let $J$ be any closed interval of length $1 - \frac{3}{2}\eta$, disjoint from $[x_0-\eta, x_0+\eta]$. Put $S := J \cup [x_0-\eta, x_0+\eta]$. Then $S\in\Seg$, and there is no $S_1\in\Seg$ with $S_1 \subseteq S$, disjoint from $A$. Hence, $A\notin\MB(\Seg)$.
\end{proof}

\begin{prop}
$\NULL$ is $\MBC$.
\end{prop}

\begin{proof}
Again, define $\Seg$ as the family of all finite unions of closed intervals with rational endpoints, which have measure greater than $1$, and let $b\colon \N \to \Q$ be a fixed bijection.
Define 
$$\calF := \{b^{-1}[S\cap\Q] :\ S\in\Seg\}.$$
We will check that $\MB(\calF) = \NULL$.\\
Take $A\in \MB(\calF)$ and choose $S\in\Seg$. Then there exists $S_1\in\Seg$ such that $S_1\subseteq S$ and $A\cap b^{-1}[S_1\cap\Q] = \emptyset$. Hence, $b[A] \cap S_1 = \emptyset$. Thus, $b[A]\in\INULL$, so $A\in\NULL$.\\
On the other hand, take $A\in \NULL$ and choose $S\in\Seg$. As $b[A]\in\INULL$, there exists $S_1\in \Seg$ such that $S_1\subseteq S$ and $b[A]\cap S_1 = \emptyset$. Thus, $A \cap b^{-1}[S_1\cap\Q] = \emptyset$, which proves that $A\in \MB(\calF)$.
\end{proof}


Suppose that an ideal $\mathcal{J}\subseteq \mathcal{P}(\N)$ has a topological representation, i.e., there exists a dense countable set $D\subseteq 2^{\N}$, a $\sigma$-ideal $I$ on $2^{\N}$, containing all singletons, and a bijection $f\colon\N\to D$ such that $A\in\J \Leftrightarrow \cl(f[A])\in I$. Then we may consider a subideal $\J_c \subseteq \J$ defined as:
$$\J_c :=\left\{A\subseteq \N :\ |\cl(f[A])| \leq \omega\right\}.$$
Call such an ideal $\J_c$ \emph{countably topologically representable}.

\begin{problem}
Find (an inner) characterization of such ideals.
\end{problem}

\begin{problem}
Suppose that $f\colon\N\to\Q$ is any bijection. Is 
$$\J_c :=\left\{A\subseteq \N :\ |\cl(f[A])| \leq \omega\right\}$$
an $\MBC$ ideal?
\end{problem}

  
\section{$\finbw$ property and extendability to summable ideals}		



In the article \cite{H1}, R. Filip\'ow, N. Mro\.zek, I. Rec\l{}aw, and P. Szuca introduced a Bolzano-Weierstrass property for ideals. This property is a generalization of a well-known classical Bolzano-Weierstrass theorem.

\begin{df}[\cite{H1}]
We say that an ideal $\I$ on $\N$ has the \emph{$\finbw$ property} if for every bounded sequence $(x_n)_{n\in\N}$ of real numbers there exists $A\notin\I$ such that the subsequence $(x_n)_{n\in A}$ is convergent.
\end{df}


Let us now introduce some notations concerning partitions of a set. 

By $\FinPart$ we denote the family of all finite partitions of $\N$ (i.e., partitions into finitely many parts). For $\mathcal{P}, \mathcal{R} \in \FinPart$, we say that \emph{$\mathcal{P}$ is finer than $\mathcal{R}$} if $\forall_{A \in \mathcal{P}}\ \exists_{B \in \mathcal{R}}\ A \subseteq B$.

For an arbitrary partition $\mathcal{P}$ of $\N$ let us denote: 
$$H^{*}(\mathcal{P}) := \{A\subseteq\N :\ \exists_{B\in\mathcal{P}}\ A\subseteq^{*} B\},$$ 
where $A\subseteq^{*} B$ is the standard relation of "almost inclusion", i.e., $A\setminus B$ is a finite set.\\

The authors of \cite{BFMS11} proved the following characterization of ideals with the property $\finbw$:

\begin{prop}[{\cite[Proposition 3]{BFMS11}}] \label{tree-fin-bw}
An ideal $\I$ does not have the $\finbw$ property if and only if there exists a sequence $(\mathcal{P}_n)_{n\in\N}$ of finite partitions of $\N$, such that each $\mathcal{P}_{n+1}$ is finer than $\mathcal{P}_n$, and such that if $(A_n)_{n\in\N}$ is a decreasing sequence with $A_n \in \mathcal{P}_n$ for each $n$, and a set $Z\subseteq\N$ is such that $Z\setminus A_n \in\Fin$ for each $n$, then $Z\in\I$.
\end{prop}

We show that the above characterization is equivalent to the following:

\begin{prop}
An ideal $\I$ does not have the $\finbw$ property if and only if there exists a sequence $(\mathcal{P}_n)_{n\in\N}$ of finite partitions of $\N$, such that: 
$$\bigcap_{n\in\N}{H^{*}(\mathcal{P}_n)}\subseteq\I.$$
\end{prop}

\begin{proof}
Assume that $\I$ does not have the $\finbw$ property, and let $(\mathcal{P}_n)_{n\in\N}$ be a suitable sequence of finite partitions of $\N$, such that each $\mathcal{P}_{n+1}$ is finer than $\mathcal{P}_n$ (by Proposition \ref{tree-fin-bw}). Let $Z \in \bigcap_{n\in\N}{H^{*}(\mathcal{P}_n)}$. If $Z\in\Fin$, then the proof is finished. Thus, assume that $Z$ is infinite. For each $n\in\N$ choose $A_n\in \mathcal{P}_n$ such that $Z \subseteq^* A_n$ (notice that such $A_n$ is unique by virtue of the fact that $\mathcal{P}_n$ is a partition). Then, $A_{n+1} \subseteq A_n$. Indeed, suppose that it is not true -- then $A_{n+1} \cap A_n = \emptyset$ (since $\mathcal{P}_{n+1}$ is finer than $\mathcal{P}_n$), which is a contradiction with $Z \subseteq^* A_n$ and $Z \subseteq^* A_{n+1}$. Hence, by Proposition \ref{tree-fin-bw}, $Z\in\I$, so $\bigcap_{n\in\N}{H^{*}(\mathcal{P}_n)} \subseteq \I$.

To prove the converse implication, assume that $(\mathcal{P}_n)_{n\in\N}$ is a sequence of finite partitions of $\N$, such that $\bigcap_{n\in\N}{H^{*}(\mathcal{P}_n)} \subseteq \I$. Define 
$$\mathcal{R}_n := \bigsqcup_{i=1}^{n}{\mathcal{P}_i} = \left\{\bigcap_{i=1}^{n}{P_i} :\ \forall_{i=1,\ldots,n}\ P_i\in\mathcal{P}_i,\ \bigcap_{i=1}^{n}{P_i}\neq\emptyset\right\}.$$
Note that then each $\mathcal{R}_{n+1}$ is finer than $\mathcal{R}_n$.
Suppose that $Z$ is such that there exist $A_n \in \mathcal{R}_n$ with $A_{n+1} \subseteq A_n$ and with $Z \subseteq^* A_n$ for every $n$. Then for each $n\in\N$ there exists $B_n \in \mathcal{P}_n$ such that $A_n \subseteq B_n$, and therefore for each $n\in\N$ we have $Z \in H^{*}(\mathcal{P}_n)$. Hence, by our assumption, we obtain that $Z\in\I$.
\end{proof}

With the use of this characterization we have obtained an interesting result for the ideals of Furstenberg, Golomb, and Kirch:

\begin{thm}
The ideals $\I_F$, $\I_G$, and $\I_K$ do not have the $\finbw$ property.
\end{thm}

\begin{proof}
Let the sequence $(\mathcal{P}_k)_{k\in\N}$ of finite partitions of $\N$ be defined as follows:
$$\mathcal{P}_1 := \{\arithseq{n}{1}\}$$
$$\mathcal{P}_2 := \{\arithseq{2}{1}, \arithseq{2}{2}\}$$
$$\mathcal{P}_3 := \{\arithseq{3}{1}, \arithseq{3}{2}, \arithseq{3}{3}\}$$
$$\vdots$$
$$\mathcal{P}_k := \{\arithseq{k}{1}, \arithseq{k}{2}, \ldots, \arithseq{k}{k}\}$$
$$\vdots$$

Then, for every $k\in\N$, 
$$H^{*}(\mathcal{P}_k)= \{A\subseteq\N :\ \exists_{b_k\in\{1,2,\ldots,k\}}\ A\subseteq^* \arithseq{k}{b_k}\}.$$
Thus,
$$\bigcap_{k\in\N}{H^{*}(\mathcal{P}_k)}= \{A\subseteq\N :\ \forall_{k\in\N}\ \exists_{b_k\in\{1,2,\ldots,k\}}\ A\subseteq^* \arithseq{k}{b_k}\}.$$
We will check that $\bigcap_{k\in\N}{H^{*}(\mathcal{P}_k)}\subseteq\I_F$. Choose any set $A\subseteq\N$ such that: 
$$\forall_{k\in\N}\ \exists_{b_k\in\{1,2,\ldots,k\}}\ A\subseteq^* \arithseq{k}{b_k}.$$ 
We need to show that $A$ is nowhere dense in the Furstenberg's topology, i.e., that for every $\arithseq{a}{b}\in \B_F$ there exists $\arithseq{c}{d}\subseteq \arithseq{a}{b}$ with $\arithseq{c}{d}\in \B_F$, such that $A\cap \arithseq{c}{d} = \emptyset$. Fix any $\arithseq{a}{b}\in \B_F$ (we then know that $b\leq a$). Then there exists $b_a\in\{1,2,\ldots,a\}$ such that $A\subseteq^* \arithseq{a}{b_a}$.\\ 
Let us consider the two cases:
\begin{itemize}
	\item[(i)] If $b_a\neq b$, then $\arithseq{a}{b} \cap \arithseq{a}{b_a} = \emptyset$, so $\arithseq{a}{b}\cap A \in\Fin$. 
	As the Furstenberg's topology is Hausdorff, $\arithseq{a}{b}$ without those finitely many points is open and disjoint from $A$, hence it contains a basic set $\arithseq{c}{d}$ disjoint from $A$. 
	\item[(ii)] If $b_a = b$, then $A\subseteq^* \arithseq{a}{b}$. Notice that $\arithseq{a}{b}=\arithseq{2a}{b}\cup \arithseq{2a}{b + a}$ (it follows from the splitting property of $\B_F$). By our assumption, there exists $b_{2a}\leq 2a$ such that $A\subseteq^* \arithseq{2a}{b_{2a}}$. Now, if $b_{2a} \neq b$, we proceed like in the case (i). If however $b_{2a}=b$, then we observe that $b_{2a}\neq b+a$, and, again, we proceed like in the case (i).
\end{itemize}

The proof for $\I_G$ and $\I_K$ goes similarly.
\end{proof}


Following K. Mazur (\cite{Maz}), let us now present the notion of a summable ideal.

\begin{df}[\cite{Maz}]
An ideal $\I$ is called a \emph{summable ideal} if there is a divergent series $\sum_{n}{a_n}$ of nonnegative reals, such that $\I=\left\{A\subseteq \N :\ \sum_{n\in A}{a_n} <\infty\right\}$.
\end{df}

The most important and well-known example of a summable ideal is the ideal:  
$$\I_{\frac{1}{n}} := \left\{A\subseteq \N :\ \sum_{n\in A}{\frac{1}{n}} <\infty\right\}.$$

By many mathematicians (e.g., \cite{Au}, \cite{FreedSem}, or, recently, \cite{Klinga}) the property of \emph{extendability to summable ideals} has been studied. It can be connected to the Riemann's rearrangement theorem (i.e., the theorem saying that if a series is conditionally convergent, then its terms can be rearranged so that the new series converges to any given value). In \cite{W}, Wilczy\'nski strengthened the Riemann's result by showing that it is enough to rearrange terms whose indices form a set of asymptotic density zero. He also posed a problem about giving a characterization of all ideals having the analogous property, i.e., ideals $\I$ such that for every conditionally convergent series $\sum_n{a_n}$ and for any $r\in\R$ there exists a permutation $\sigma \colon \N\to\N$ such that $\sum_n{a_{\sigma(n)}} = r$ and $\{n :\ \sigma(n)\neq n\}\in\I$. We say that such ideals have the \emph{Riemann property}.

In the paper \cite{H3}, R. Filip\'ow and P. Szuca solved this problem by proving that:
\begin{thm}[{\cite[Theorem 3.3]{H3}}]
An ideal on $\N$ has the Riemann property if and only if it cannot be extended to a summable ideal.
\end{thm}

Considering the examples mentioned in Section \ref{examples}, one can easily conclude that the ideals $\I_F$, $\I_G$, and $\I_K$ are not contained in the summable ideal $\I_{\frac{1}{n}}$ -- since neither the set of primes $\mathbb{P}$ nor the set of even numbers $\arithseq{2}{2}$ belong to $\I_{\frac{1}{n}}$, the Example \ref{primes} proves that $\I_F \not\subseteq \I_{\frac{1}{n}}$, while the Example \ref{even} proves that $\I_G \not\subseteq \I_{\frac{1}{n}}$ and $\I_K \not\subseteq \I_{\frac{1}{n}}$.
This may indicate that none of these ideals can be extended to a summable ideal.

The result from the paper \cite{H3} allows to confirm these presumptions.

\begin{prop}[{\cite[Corollary 3.5]{H3}}]
If an ideal does not have the $\finbw$ property, then it cannot be extended to a summable ideal.
\end{prop}

We obtain an immediate corollary:
\begin{cor}
The ideals $\I_F$, $\I_G$, and $\I_K$ cannot be extended to summable ideals (hence, they have the Riemann property).
\end{cor}



\begin{thebibliography}{abc}

\bibitem{Au}
Auerbach H., \emph{$\ddot{\textrm{U}}$ber die Vorzeichenverteilung in unendlichen Reihen.},
Studia Math. {\bf 2} (1930) 228--230.

\bibitem{MB}
Balcerzak M., Bartoszewicz A., Rzepecka J., Wro\'nski S., \emph{Marczewski fields and ideals},
Real Anal. Exchange {\bf 26}(2) (2001) 703--715.

\bibitem{MB2}
Balcerzak M., Bartoszewicz A., Ciesielski K., \emph{Algebras with inner MB-representation},
Real Anal. Exchange {\bf 29}(1) (2004) 265--274.

\bibitem{MB3}
Balcerzak M., Bartoszewicz A., Ciesielski K., \emph{On Marczewski-Burstin representations of certain algebras of sets},
Real Anal. Exchange {\bf 26} (2001) 581--592.

\bibitem{MB4}
Balcerzak M., Rzepecka J., \emph{On Marczewski-Burstin representations of algebras and ideals},
J. Appl. Anal. {\bf 9}(2) (2003) 275--286.

\bibitem{BFMS11}
Barbarski P., Filip\'ow R., Mro\.zek N., Szuca P., \emph{Uniform density $u$ and $\I_u$-convergence on a big set},
Math. Commun. {\bf 16}(1) (2011) 125--130.

\bibitem{B}
Brown M., \emph{A countable connected Hausdorff space},
In: Cohen L.M., The April Meeting in New York, Bull. Amer. Math. Soc. {\bf 59}(4) (1953) 367.

\bibitem{FS}
Farah I., Solecki S., \emph{Two $F_{\sigma\delta}$ ideals},
Proc. Amer. Math. Soc. {\bf 131}(6) (2003) 1971--1975.

\bibitem{H1}
Filip\'ow R., Mro\.zek N., Rec\l{}aw I., Szuca P., \emph{Ideal convergence of bounded sequences},
J. Symbolic Logic {\bf 72}(2) (2007) 501--512.

\bibitem{H3}
Filip\'ow R., Szuca P., \emph{Rearrangement of conditionally convergent series on a small set},
J. Math. Anal. Appl. {\bf 362}(1) (2010) 64--71.

\bibitem{F}
Furstenberg H., \emph{On the infinitude of primes},
Amer. Math. Monthly {\bf 62}(5) (1955) 353.

\bibitem{G}
Golomb S.W., \emph{A connected topology for the integers},
Amer. Math. Monthly {\bf 66}(8) (1959) 663--665.

\bibitem{Kechris}
Kechris A.S., \emph{Classical Descriptive Set Theory},
Grad. Texts in Math. {\bf 156}, Springer-Verlag, New York, 1995.

\bibitem{K}
Kirch A.M., \emph{A countable, connected, locally connected Hausdorff space},
Amer. Math. Monthly {\bf 76}(2) (1969) 169--171.

\bibitem{Klinga}
Klinga P., Nowik A., \emph{Extendability to summable ideals},
Acta Math. Hungar. {\bf 152}(1) (2017) 150--160.

\bibitem{Adas}
Kwela A., Sabok M., \emph{Topological representations},
J. Math. Anal. Appl. {\bf 422} (2015) 1434--1446.

\bibitem{KwelaZak}
Kwela A., Zakrzewski P., \emph{Combinatorics of ideals -- selectivity versus density}, unpublished extended version available at:
http://kwela.strony.ug.edu.pl/papers/Combinatorics\underline{ }of\underline{ }ide-als\underline{ }extended.pdf,
last accessed May 31st, 2018.

\bibitem{Maz}
Mazur K., \emph{$F_\sigma$-ideals and $\omega_1\omega_1^*$-gaps in the Boolean algebras $\mathcal{P}(\omega)/\mathcal{I}$},
Fund. Math. {\bf 138}(2) (1991) 103--111.

\bibitem{Sabok} 
Sabok M., Zapletal J., \emph{Forcing properties of ideals of closed sets}, 
J. Symbolic Logic {\bf 76}(3) (2011) 1075--1095.

\bibitem{FreedSem}
Semer J.J., Freedman A.R., \emph{On summing sequences of 0's and 1's},
Rocky Mountain J. Math. {\bf 11}(3) (1981) 419--425.

\bibitem{Szczuka1} 
Szczuka P., \emph{Connections between connected topological spaces on the set of positive integers}, 
Cent. Eur. J. Math. {\bf 11}(5) (2013) 876--881.

\bibitem{Szczuka2}
Szczuka P., \emph{Properties of the division topology on the set of positive integers},
Int. J. Number Theory {\bf 12}(3) (2016) 775--785.

\bibitem{Szczuka3}
Szczuka P., \emph{The closures of arithmetic progressions in the common division topology on the set of positive integers},
Cent. Eur. J. Math. {\bf 12}(7) (2014) 1008--1014.

\bibitem{Szczuka4}
Szczuka P., \emph{The connectedness of arithmetic progressions in Furstenberg's, Golomb's and Kirch's topologies},
Demonstratio Math. {\bf 43}(4) (2010) 899--909.

\bibitem{Sz}
Szpilrajn (Marczewski) E., \emph{Sur une classe de fonctions de M. Sierpi\'nski et la classe correspondante d'ensembles},
Fund. Math. {\bf 24} (1935) 17--34.

\bibitem{W}
Wilczy\'nski W., \emph{On Riemann derangement theorem},
S\l{}upskie Prace Matematyczno-Fizyczne {\bf 4} (2007) 79--82.

\end{thebibliography}

\end{document}
