\documentclass{amsart}

\usepackage[mathscr]{euscript}
\usepackage{amssymb}
\usepackage{amsmath}
\usepackage{latexsym}

%\usepackage{tikz}
%\usetikzlibrary{matrix}

%-----------------
\usepackage{xcolor}
\usepackage{polski}
\usepackage[utf8]{inputenc}
%-----------------

\makeatletter
\renewcommand\@biblabel[1]{#1.}
\makeatother
\newtheorem{thm}{Theorem}
\newtheorem{lem}{Lemma}
\newtheorem{prop}{Proposition}
\newtheorem{fact}{Fact}
\newtheorem{cor}{Corollary}
\theoremstyle{definition}
\newtheorem{problem}{Problem}
\newtheorem{question}{Question}
\newtheorem{df}{Definition}
\newtheorem{remark}{Remark}
\theoremstyle{definition}
\newtheorem{ex}{Example}
\newtheorem{conj}{Conjecture}
\newtheorem{observation}{Observation}

\newcommand{\N}{{\mathbb N}}
\newcommand{\Z}{{\mathbb Z}}
\newcommand{\R}{{\mathbb R}}
\newcommand{\Q}{{\mathbb Q}}
\newcommand{\Fin}{\textrm{Fin}} 
\newcommand{\Ctbl}{\textrm{Ctbl}} 
\newcommand{\ce}{\mf c}
\newcommand{\eps}{\varepsilon}
\newcommand{\I}{\mathcal I}
\newcommand{\J}{\mathcal J}
\newcommand{\h}{\mathcal H}
\newcommand{\T}{\mathcal{T}}
%\newcommand{\SqrFr}{\textrm{SqrFr}}
\newcommand{\SqrFr}{\mathbb{SF}}
\newcommand{\calF}{\mathcal{F}}
\newcommand{\calK}{\mathcal{K}}

\newcommand{\bw}{\text{BW}}
\newcommand{\hbw}{\text{hBW}}
\newcommand{\finbw}{\text{FinBW}}
\newcommand{\hfinbw}{\text{hFinBW}}
\DeclareMathOperator{\Exists}{\exists}
\DeclareMathOperator{\Forall}{\forall}
%%% odrobinka oznaczen dotyczacych partycji:
\newcommand{\Partitions}{(\omega)^{\leq \omega}}
\newcommand{\InfPart}{(\omega)^{\omega}}
\newcommand{\FinPart}{(\omega)^{< \omega}}

\title[Ideals of nowhere dense sets in some topologies on integers]{Ideals of nowhere dense sets in some topologies on integers}

% on natural numbers??/ positive integers (szczuka)?

\author{Marta Kwela}
\address{Marta Kwela, Institute of Mathematics, Faculty of Mathematics, Physics and Informatics, University of Gda\'{n}sk, ul.~Wita Stwosza 57, 80-308 Gda\'{n}sk, Poland}
\email{Marta.Kwela@mat.ug.edu.pl}

\author{Andrzej Nowik}
\address{Andrzej Nowik, Institute of Mathematics, Faculty of Mathematics, Physics and Informatics, University of Gda\'{n}sk, ul.~Wita Stwosza 57, 80-308 Gda\'{n}sk, Poland}
\email{andrzej@mat.ug.edu.pl}

\begin{document}
\begin{abstract}
We investigate the ideals of nowhere dense sets in three topologies on $\N$, related to arithmetic progressions. In particular, we explore relationships between those ideals and show that each of them has a \textit{topological representation}.
\end{abstract}
\maketitle



%\section{Wstęp}

%W ramach doktoratu zajmuję się badaniem ideałów topologicznych na zbiorze liczb naturalnych, związanych z ciągami arytmetycznymi. Drugi rozdział jest poświęcony wprowadzeniu w te zagadnienia oraz definiuje pojęcia używane w następnych częściach projektu. Rozdział trzeci zawiera opis dotychczas uzyskanych wyników wspólnej pracy z prof. A. Nowikiem (\cite{M}), związanych m.in. z pojęciami ideałów reprezentowanych topologicznie oraz własności $\finbw$ dla ideałów i rozszerzalności do ideałów sumowalnych. W ostatnim rozdziale określone zostały problemy, którymi będę chciała się zajmować w najbliższej przyszłości, w szczególności dotyczące reprezentacji Marczewskiego-Burstina oraz jednorodności i porządków na ideałach.


\section{Preliminaries}

Let $\N$ denote the set of positive integers and $\N_0$ -- the set of non-negative integers. For all $a,b\in\N$ the symbol $\{an+b\}$ stands for the infinite arithmetic progression with the initial term $b$ and the difference $a$:
$$\{an+b\} = \{an+b\ :\ n\in\N_0\} = \{b,\ b+a,\ b+2a,\ ...\}. $$
We use the symbol $(a,b)$ to denote the greatest common divisor of $a$ and $b$. Moreover, the letter $\mathbb{P}$ symbolizes the set of all prime numbers.

By $\mathbb{SF}$ let us denote the set of \emph{square-free numbers} (i.e., numbers not divisible by any square greater than 1):
%i.e.: $\{n\colon \forall_{k > 1} \neg (k^2 | n)\}$.
$$\SqrFr = \{1,2,3,5,6,7,10,11,\ldots\}.$$
By \emph{squareful numbers} we mean numbers which are not square-free, i.e., numbers for which at least one prime factor exponent is 2.

 If $X, Y \in \Partitions$ then we say that $Y$ is
{\it coarser} than $X$ iff
  $\forall_{A \in X} \exists_{B \in Y} (A \subseteq B)$
and we write $Y \sqsubseteq X$.

% ozn. [omega]^omega...?

% base/basis of/for topology???

\section*{Three topologies}

For $a,b\in\N$ one can consider three topologies on $\N$:
\begin{itemize}
\item \emph{Furstenberg's topology} $\T_F$ with the base $\mathcal{B}_F = \{\{an+b\}\ :\ b\leq a\}$;
\item \emph{Golomb's topology} $\T_G$ with the base $\mathcal{B}_G = \{\{an+b\}\ :\ (a,b)=1,\ b<a\}$;
\item \emph{Kirch's topology} $\T_K$ with the base $\mathcal{B}_K = \{\{an+b\}\ :\ (a,b)=1,\ b<a,\ a\in\SqrFr\}$.
\end{itemize}

The topology $\T_F$ was introduced in 1955 by H. Furstenberg in \cite{F}. With its use, he presented an elegant topological proof of the existence of infinitely many prime numbers. In 1959 S. Golomb in \cite{G} presented a similar proof using the $\T_G$ topology defined in 1953 by M. Brown in \cite{B}. In 1969, A. Kirch in \cite{K} defined a topology $\T_K$, weaker than the topology of Golomb. All of these topologies have recently been studied by P. Szczuka, e.g. in \cite{Szczuka1}, \cite{Szczuka2}, \cite{Szczuka3}.

F: \textcolor{gray}{\scriptsize{[normal, metrizable, zero-dimensional, totally disconnected]}}
G: \textcolor{gray}{\scriptsize{[Hausdorff but not regular, connected, not locally connected]}}
K: \textcolor{gray}{\scriptsize{[Hausdorff but not regular, connected, locally connected]}}

%%%// Furstenberg - na $\Z$, ..., ale my na $\N$ dla ujednolicenia; 
%%%przy założeniu $b<=a$...
% has been introduced??

\section*{Three ideals}

An \emph{ideal} on $\N$ is a family of subsets of $\N$ closed under taking finite unions and subsets of its elements. We assume that an ideal is proper ($\neq \mathcal{P}(\N)$) and contains all finite sets.
%We say that a set $A$ is \emph{nowhere dense in topology $\T$} if its closure has empty interior, or, equivalently:
%$$\forall_{U\in\T}\ \exists_{V\in\T,\ V\subseteq U}\ A\cap V = \emptyset.$$

%ideały są \emph{właściwe}, ($\neq \mathcal{P}(\N)$, czyli $\N$ nie należy do ideału), oraz że zawierają wszystkie skończone podzbiory $\N$.

%%%\textcolor{red}{nie w każdej topologii zawierają Fina...!}

Suppose that $b\colon\N\to\Q$ is any fixed bijection.
Let us recall that by $\mathrm{NDW}$ we denote the ideal of sets
$A \subset \N$ such that $\mathit{cl}(\lbrace b(n)\colon n \in A\rbrace)$
is a nowhere dense set.

Obviously, in any topology, the nowhere dense sets form an ideal. Let us then define three ideals on $\N$:
\begin{itemize}
\item \emph{Furstenberg's ideal} $\I_F$ of all nowhere dense sets in $\T_F$;
\item \emph{Golomb's ideal} $\I_G$ of all nowhere dense sets in $\T_G$;
\item \emph{Kirch's ideal} $\I_K$ of all nowhere dense sets in $\T_K$.
\end{itemize}

%takie ideały nazywamy \emph{ideałami topologicznymi} (zob. \cite{MB})

\section*{"Splitting property"}

% [AN] pewnie jakies zagajenie by sie tu przydalo...
%%%splitting property - jak w dowodach przykładów... 
% [AN] i pewnie trzeba w ogole przeniesc ten rozdzial za rozdzial o MB
% gdyz tutaj pojecia MB reprezentacji uzywamy!

% IF,G,K in Fsd
% bazy mają własność "przeliczalnego splittingu"
% IF,G,K tall
% bazy mają własność "splittingu"
% IF,G,K notin Fs

Let us say that a family $\mathcal{F} \subseteq [\omega]^\omega$ has the \emph{splitting property} if and only if for any $F \in \mathcal{F}$ one can find $F_1, F_2 \in \mathcal{F}$ such that $F_1 \cup F_2 \subseteq F$ and $F_1 \cap F_2 = \emptyset$. Notice that if a family $\mathcal{F}$ has the splitting property, then it also has a \emph{countable splitting property}, i.e., there exists an infinite countable family $\mathcal{G} \subseteq \mathcal{F}$ such that $\bigcup \mathcal{G} \subseteq \mathcal{F}$ and the family $\mathcal{G}$ is pairwise disjoint.

\begin{observation}[Crucial observation]
All families: $\mathcal{B}_F$, $\mathcal{B}_G$ and $\mathcal{B}_K$ have the splitting property.
\end{observation}
\begin{proof}
For the case of the Furstenberg's topology it suffices to observe that $\{2an + b\} \subseteq \{an + b\}$, $\{2an + a + b\} \subseteq \{an + b\}$, and $\{2an + b\} \cap \{2an + a + b\} = \emptyset$.



\begin{center}
\begin{picture}(260,180)
\put(0,0){\makebox(0,0){$\vdots$}}
\put(0,10){\makebox(0,0){$\{16an+b\}$}}
\put(100,10){\makebox(0,0){$\{16an+8a+b\}$}}
\put(50,50){\makebox(0,0){$\{8an+b\}$}}
\put(65,40){\vector(1,-1){20}}
\put(35,40){\vector(-1,-1){20}}
\put(150,50){\makebox(0,0){$\{8an+4a+b\}$}}
\put(100,90){\makebox(0,0){$\{4an+b\}$}}
\put(115,80){\vector(1,-1){20}}
\put(85,80){\vector(-1,-1){20}}
\put(200,90){\makebox(0,0){$\{4an+2a+b\}$}}
\put(150,130){\makebox(0,0){$\{2an+b\}$}}
\put(165,120){\vector(1,-1){20}}
\put(135,120){\vector(-1,-1){20}}
\put(250,130){\makebox(0,0){$\{2an+a+b\}$}}
\put(200,170){\makebox(0,0){$\{an+b\}$}}
\put(215,160){\vector(1,-1){20}}
\put(185,160){\vector(-1,-1){20}}
\end{picture}
\end{center}
\vspace{0.5cm}

A w przypadku bazy dla topologii Golomba czy Kircha?\\
Prawdopodobnie też da się przeprowadzić poprzednią konstrukcję "`splittingu"', lecz odpowiednio ją modyfikując.\\
Np. dla topologii Golomba (czyli zakładamy, że $(a,b)=1$):\\
Dobieramy $p\in Primes$ t.że $p\nmid b$ (+ dodatkowo by $p\nmid a+b$).

\begin{center}
\begin{picture}(110,60)
\put(0,0){\makebox(0,0){$\vdots$}}
\put(0,10){\makebox(0,0){$\{pan+b\}$}}
\put(100,10){\makebox(0,0){$\{pan+a+b\}$}}
\put(50,50){\makebox(0,0){$\{an+b\}$}}
\put(65,40){\vector(1,-1){20}}
\put(35,40){\vector(-1,-1){20}}
\end{picture}
\end{center}
\vspace{0.5cm}

(rozłączne) Gdyby $pan_1+b=pan_2+a+b$, to $pa(n_1-n_2)=a$, więc $p(n_1-n_2)=1$, sprzeczność.
$(a,b)=1 \Rightarrow (pa,b)=1$\\
$(a,b)=1 \Rightarrow (a,a+b)=1 \Rightarrow (pa,a+b)=1$\\
Stosując rozumowanie do $\{pan+b\}$, otrzymujemy odpowiedni splitting.\\

Dla topologii Kircha można przeprowadzić identyczne rozumowanie, tylko trzeba oprócz założenia, że $p\nmid b$, $p\nmid a+b$ założyć też jeszcze, że $p\nmid a$ (wówczas skoro $a\in \SqrFr$, to $pa\in \SqrFr$).

%% Ale dokładnie partycji się zrobić nie da??
\end{proof}


%\textbf{\underline{Podsumowanie:}} $\mathcal{B}_F$, $\mathcal{B}_G$ i $\mathcal{B}_K$ mają własność "`splittingu"', czyli:
%$$\forall_{F\in\mathcal{B}} \exists_{G_1, G_2 \in\mathcal{B}} G_1, G_2 \subseteq F, G_1\cap G_2 =\emptyset.$$ \\

%\textbf{\underline{Wniosek:}} Bazy te mają własność "przeliczalnego splittingu", czyli:
%$$\forall_{F\in\mathcal{B}} \exists_{G_n \in\mathcal{B}} \left[ \forall_n G_n \subseteq F, \forall_{n_1\neq n_2} G_{n_1}\cap G_{n_2}=\emptyset \right].$$ \\
%\exists_{(G_n)_n \subseteq\mathcal{B}}

\begin{remark} \label{remH}
The base of any Hausdorff topology $\T$ has the splitting property.
\end{remark}
\begin{proof}
Let $\mathcal{B}$ be the base of the topology $\T$. Take any $B\in\mathcal{B}$ and two different points $x,y\in B$. As $\T$ is Hausdorff, there exist two open sets $V,W\in\T$ such that $x\in V$, $y\in W$, and $V\cap W = \emptyset$. Let $V'=B\cap V$, $W'=B\cap W$. $V'$ and $W'$ are open, so each contains a basic set: $B_{V'}$ and $B_{W'}$, respectively. Now it is clear that $B_{V'}\cup B_{W'}\subseteq B$ and $B_{V'}\cap B_{W'}=\emptyset$.
\end{proof}
%If the base of a Hausdorff topology $\T$ consists only of infinite sets, then it has the splitting property. - chyba to założenie niepotrzebne?...




\section{General properties}
% basic properties?

%\begin{theorem}
%$\I_F$, $\I_G$ i $\I_K$ są ideałami typu $F_{\sigma\delta}$, ale nie $F_{\sigma}$.
%\end{theorem}

%%   see:/cf.?

\begin{ex} 
The set $A = \{n! \ :\ n\in\N\}$ belongs to $\I_F$, $\I_G$ and $\I_K$.
\end{ex}

\begin{proof}
%[using splitting property...]
We need to show that for any $\{an+b\} \in \mathcal{B}_F$ there exists $\{cn+d\} \in \mathcal{B}_F$ with $\{cn+d\} \subseteq \{an+b\}$, such that $\{cn+d\}\cap A = \emptyset$. Let us fix $\{an+b\} \in \mathcal{B}_F$. Note that in the set $A$ all but finitely many elements (for $n\geq a$) are divisible by $a$. Consider the two cases:
\begin{itemize}
 \item[(i)] $b\neq a$. Set $s:= \min \{n\ :\ an+b>a!\}$. Take $\{cn+d\} = \{a(s+1)n+as+b\} \subseteq \{an+b\}$. Then, $as+b\leq as+a = a(s+1)$, so $\{cn+d\}\in \mathcal{B}_F$. Moreover, $as+b>a!$ and all elements of $\{cn+d\}$ are not divisible by $a$, so $\{cn+d\}\cap A = \emptyset$.
 \item[(ii)] $b=a$. Then, $\{an+b\} = \{an+a\}$ and we can split it into two disjoint subsets: $\{an+a\} = \{2an+a\}\cup \{2an+2a\}$. In the set $A$ all but finitely many elements (for $n\geq 2a$) are divisible by $2a$, so almost all elements belong to $\{2an+2a\}$. Set $s:= \min \{n\ :\ 2an+a>(2a)!\}$. Take $\{cn+d\} = \{2a(s+1)n+2as+a\} \subseteq \{an+a\}$. Then, $2as+a\leq 2as+2a = 2a(s+1)$, so $\{cn+d\}\in \mathcal{B}_F$. Moreover, $2as+a>(2a)!$ and all elements of $\{cn+d\}$ are not divisible by $2a$, so $\{cn+d\}\cap A = \emptyset$.
\end{itemize}
Note that instead we could have used the fact that Furstenberg's topology is Hausdorff. The crucial observation is that all but finitely many elements from the set $A$ are divisible by some constant $a$. Since singleton sets are closed, $\{an+b\}$ without finitely many points is open and disjoint from $A$ for $b<a$, hence it contains a basic set disjoint from $A$. Case $(ii)$ uses the splitting property, which also follows from $\T_F$ being Hausdorff (see: Remark \ref{remH}).\\
Thus, the proof for $\I_G$ and $\I_K$ (both $\T_G$ and $\T_K$ are also Hausdorff) goes similarly (in basic sets of these topologies we assume that $b<a$, so we only need to consider case $(i)$).
\end{proof}

%Poniższe proste przykłady pokazują, że ideał Furstenberga znacząco różni się od ideałów Golomba i Kircha.
%[AN] przetlumaczone na to ponizej, nie wiadomo czy skladnie?
We will construct examples which 
distinguish these ideals and prove some
inclusions among some of these ideals, namely
%%%show that all these ideals are different, namely
%examples shows that in some way the Furstenberg's ideal significantly differs from the other ideals (namely: from the Golomb's and the Kirch's ideal.)



\begin{ex}[\cite{Szczuka4}] 
$\mathbb{P}\in \I_F$, but $\mathbb{P}$ is dense in $\T_G$ and $\T_K$ (therefore it does not belong to $\I_G$ nor $\I_K$).
\end{ex}

\begin{ex} 
$\SqrFr\in \I_F$, but $\SqrFr$ is dense in $\T_G$ and $\T_K$ (therefore it does not belong to $\I_G$ nor $\I_K$).
\end{ex}

\begin{proof}
Firstly, note that the set of squareful numbers is open in $\T_F$ as it is equal to the sum $\bigcup_{p\in\mathbb{P}} \{p^2 n+p^2\}$ of arithmetic progressions belonging to $\mathcal{B}_F$. Thus, the set $\SqrFr$ is a complement of an open set, so it is closed (and hence its closure is equal to itself). It suffices to show that $\SqrFr$ has empty interior.

Let us observe that in every arithmetic progression one can find a squareful number. Indeed, in an arbitrary arithmetic progression $\{an+b\}$, for fixed $a,b\in\N$, put $n_0 = a^2 +ab+2a+2b+1$. Then,
$$an_0 +b = a(a^2 +ab+2a+2b+1)+b = a((a+1)(a+b+1)+b)+b = $$
$$= a(a+1)(a+b+1)+(a+1)b = (a+1)(a(a+b+1)+b) =$$
$$= (a+1)(a(a+1)+b(a+1))= (a+1)^2 (a+b),$$
%$$an_0 +b = a(a^2 +ab+2a+2b+1)+b = a((a+1)(a+b+1)+b)+b = $$
%$$= a(a+1)(a+b+1)+(a+1)b = (a+1)(a(a+b+1)+b) = (a+1)(a(a+1)+b(a+1))=$$
%$$= (a+1)^2 (a+b),$$
which is always a squareful number, because $a+1  \neq 1$. Hence, no arithmetic progression (and, in particular, no set from $\mathcal{B}_F$) can be contained in $\SqrFr$, which proves that $\SqrFr$ has empty interior and thus it is nowhere dense in $\T_F$.

The density of $\SqrFr$ in $\T_G$ and $\T_K$ follows from the fact that $\mathbb{P}\subseteq \SqrFr$ and $\mathbb{P}$ is dense in $\T_G$ and $\T_K$ (hence so is every its superset).
\end{proof}

\begin{ex} 
The set of even numbers $\{2n+2\}$ is in $\I_G$ and $\I_K$, but it belongs to the base of $\T_F$ (therefore $\{2n+2\}\notin \I_F$).
\end{ex}

\begin{proof}
We need to show that for any $\{an+b\} \in \mathcal{B}_G$ there exists $\{cn+d\} \in \mathcal{B}_G$ with $\{cn+d\} \subseteq \{an+b\}$, such that $\{cn+d\}\cap \{2n+2\} = \emptyset$. Let us fix $\{an+b\} \in \mathcal{B}_G$ (we assume that $(a,b)=1$ and $b<a$) and consider the two cases:
\begin{itemize}
 \item[(i)] $2\mid b$. Then, $2 \nmid a$ (otherwise, $a$ and $b$ would not be coprime). Take $\{cn+d\} = \{2an+a+b\}$. As $(a,b)=1$, we know that $(a,a+b)=1$, and as $2 \nmid a$, we have $(2a,a+b)=1$. Thus, $\{2an+a+b\}\in\mathcal{B}_G$ and $\{2an+a+b\}\subseteq\{an+b\}$. Moreover, $\{2an+a+b\}$ consists only of odd numbers, so $\{2an+a+b\}\cap \{2n+2\} = \emptyset$.
 \item[(ii)] $2\nmid b$. Take $\{cn+d\} = \{2an+b\}$. As $2\nmid b$, we have $(2a,b)=1$. Thus, $\{2an+b\}\in\mathcal{B}_G$ and $\{2an+b\}\subseteq\{an+b\}$. Moreover, $\{2an+b\}$ consists only of odd numbers, so $\{2an+b\}\cap \{2n+2\} = \emptyset$.
\end{itemize}

The proof for $\I_K$ goes similarly. In item (i) we additionally need to observe that if $2 \nmid a$ and $a$ is square-free, then $2a$ will also be square-free. In item (ii), if $2 \nmid a$, we use the same argument as above, and if $2\mid a$, we take $\{cn+d\} = \{an+b\}$ as it is already disjoint from the set of even numbers \{2n+2\}.
%Niech $E=\{2n+2\}$. Wybierzmy bazowy dla topologii G $\{an+b\}$, $(a,b)=1$. Rozważmy przypadki:
%\begin{itemize}
% \item[I] $b=2k$. Wówczas $2 \nmid a$. Niech $n=2m+1$, więc $an+b=a(2m+1)+b=2am+a+b$. Skoro $(a,b)=1$, to $(a,a+b)=1$, a także skoro $2 \nmid a$, to $(2a,a+b)=1$. Zatem $\{2am+a+b\}\in\mathcal{B}_G$ i $\{2am+a+b\}\subseteq\{an+b\}$, ponadto $\{2am+a+b\}\subseteq Odd$, czyli $\{2am+a+b\}\cap \{2n+2\} = \emptyset$.
% \item[II] $b\in Odd$. Niech $n=2m$, wówczas $an+b=2am+b$. Skoro $b\in Odd$, to $(2a,b)=1$, czyli $\{2am+b\}\in\mathcal{B}_G$ i $\{2am+b\}\subseteq\{an+b\}$. Co więcej, $2am+b\in Odd$, więc $\{2am+b\}\cap \{2n+2\} = \emptyset$.
%\end{itemize}
%Identyczny dowód "`pracuje"' w przypadku topologii Kircha, tyle że w I trzeba zauważyć, że skoro $2 \nmid a$ i $a$ jest bezkwadratowa, to $2a$ też będzie bezkwadratowa. W II, jeśli $2\mid a$, to nie musimy nic modyfikować, bo wtedy $\{an+b\}\cap \{2n+2\} = \emptyset$, jeśli zaś $2 \nmid a$, to ponownie jw. $2a$ jest bezkwadratowa.
\end{proof}

The proof of the previous result can easily be generalized for sets of multiples of any prime number, as follows:

\begin{ex} 
For any $p\in\mathbb{P}$, $\{pn+p\}\in \I_G \cap \I_K \setminus \I_F$.
\end{ex}

%\textcolor{red}{Chyba nie pisać dowodu - to oczywiste?} - [AN] nie, nie trzeba dowodu. Tak jest OK.
%\begin{proof}
%Wybierzmy bazowy dla topologii G $\{an+b\}$, $(a,b)=1$ (dla topologii K zakładamy dodatkowo, że $a$ jest bezkwadratowa). Rozważmy dwa przypadki:
%\begin{itemize}
% \item[I] $b=pk$. Wówczas $p \nmid a$. Wstawiamy $n=pm+1$ i mamy $an+b=a(pm+1)+b=pam+a+b$. Skoro $(a,b)=1$, to $(a,a+b)=1$, a skoro $p \nmid a+b$, to $(pa,a+b)=1$ (jeśli w przypadku topologii K zakładaliśmy, że $a$ jest bezkwadratowa, więc skoro $p \nmid a$, to $pa$ też jest bezkwadratowa). Mamy: $\{pam+a+b\}\subseteq\{an+b\}$ i oczywiście $\{pam+a+b\}\cap \{pn\} = \emptyset$.
% \item[II] $p \nmid b$. Podstawiamy $n=pm$, wówczas $an+b=pam+b$, $(pa,b)=1$ i $\{pam+b\}\cap \{pn\} = \emptyset$. W przypadku topologii Kircha, trzeba znów rozważyć dwa przypadki:
%\begin{itemize}
% \item[a)] $p \nmid a$. Wówczas postępujemy jw., obserwując tylko, że skoro $p \nmid a$, $a$ jest bezkwadratowa, to $pa$ też będzie bezkwadratowa.
% \item[b)] $p\mid a$. Wtedy od razu mamy $\{an+b\}\cap \{pn\} = \emptyset$.
%\end{itemize}	
%\end{itemize}
%\end{proof}

\begin{cor}
If $p_1, \ldots, p_k \in \mathbb{P}$, then $\{n\in\N\ :\ p_1\mid n \vee \ldots \vee p_k\mid n\}\in \I_G \cap \I_K \setminus \I_F$.
\end{cor}

%\begin{problem}
%Rozróżnić $\I_G$ od $\I_K$.
%\end{problem}
%Problemem pozostaje rozróżnienie ideałów Golomba i Kircha. Do tej pory udało się nam uzyskać jedynie częściowy wynik, który sugeruje, że kontrprzykładów na zawieranie tych ideałów należy szukać wśród zbiorów bardziej "skomplikowanych" niż zbiory bazowe: 

%    \color{purple}	
% [AN] OK, taki tekst moze byc, wyjasnia on dlaczego nie ma przykladu Kirch versus Golomb
%%%Distinguishing the ideals of Golomb and Kirch still remains an open problem.
The result that the ideals of Golomb and Kirch are not the same will be shown in the next theorem. So far, we only managed to obtain a result suggesting that an example of a set witnessing the lack of inclusion between those ideals cannot be found among the sets from $\mathcal{B}_G \setminus \mathcal{B}_K$:

%    \color{black}

\begin{prop}
Every set of the form $\{2sn+q\ :\ s \in\SqrFr, 2|s, (s,q)=1, q<s\}$ is in $\mathcal{B}_G$ but not in $\mathcal{B}_K$, but it does not belong to $\I_K$.
\end{prop}

\begin{proof}
Let us first observe that $\{2sn+q\}$, as defined above, satisfies: $(2s,q)=1, q<2s$ -- therefore it is in $\mathcal{B}_G$. Moreover, $2s$ is not square-free, so it must not be in $\mathcal{B}_K$.
Now, we want to show that there exists $\{an+b\}\in \mathcal{B}_K$ such that for every $\{cn+d\}\subseteq \{an+b\}$ with $\{cn+d\}\in \mathcal{B}_K$ we have: $\{2sn+q\}\cap \{cn+d\} \neq \emptyset$. Let $\{an+b\} = \{sn+q\}$ (it belongs to $\mathcal{B}_K$ since $s$ is square-free, $(s,q)=1$ and $q<s$). Take any $\{cn+d\}\subseteq \{sn+q\}$ with $\{cn+d\}\in \mathcal{B}_K$ -- we then know that $c$ is square-free. Suppose that $\{2sn+q\}\cap \{cn+d\} = \emptyset$. As $\{cn+d\}\subseteq \{sn+q\}$ and $\{sn+q\} \setminus \{2sn+q\} = \{2sn+q+s\}$, it means that $\{cn+d\}\subseteq \{2sn+q+s\}$ -- but then $2s$ must divide $c$, which hence cannot be square-free. A contradiction ends the proof.
\end{proof}
% def NWD? pusty przekrój/zawarty w dopełnieniu?/
% but -> however; must not -> cannot

\begin{thm}
$\I_K \subseteq \I_G$.
\end{thm}	   
\begin{proof}
We will show that $X \not\in I_G \implies X \not\in I_K$.
Suppose that $X \not\in \I_G$. Then
$\exists_{\{an+b\}\in \mathcal{B}_G} \forall_{\{cn+d\} \subseteq \{an+b\} \atop \{cn+d\}\in \mathcal{B}_G} \{cn+d\} \cap X \not= \emptyset$.
We have to show that 
$$\exists_{\{a^\prime n+b^\prime\}\in \mathcal{B}_K} \forall_{\{c^\prime n+d^\prime \} \subseteq \{a^\prime n+b^\prime\} \atop \{c^\prime n+d\}
\in \mathcal{B}_K} \{c^\prime n+d^\prime \} \cap X \not= \emptyset.$$
Let $a^\prime = \Pi_{p \in \Theta(a)} p$ (i.e. squarefree part of $a$)
and $b^\prime = b$. Then $\{a^\prime n+b^\prime\} \in \mathcal{B}_K$
since $a^\prime \in \SqrFr$ and $(a^\prime : b) = 1$.
Suppose that $\{c^\prime n+d^\prime \} \subseteq \{a^\prime n+b^\prime\}$,
and $\{c^\prime n+d^\prime\}\in \mathcal{B}_K$. Then $c^\prime \in \SqrFr$, $(c^\prime:d^\prime)=1$
and $a^\prime | c^\prime$. 
Observe that $(\frac{a}{a^\prime} : \frac{c^\prime}{a^\prime}) = 1$,
hence
$\{\frac{a}{a^\prime} n + b\} \cap \{\frac{c^\prime}{a^\prime} n + d^\prime \} \not= \emptyset$
(by the Chinese remainder theorem).
Define $A = \{a n+b \} \cap \{c^\prime n+d^\prime \}$, it is easy to 
conclude that $A \not= \emptyset$ hence $A$ is an arithmetic progression.
Moreover, $A = \{\frac{a c^\prime}{a^\prime} n + x\}$
for some $x \in A$.
Notice that $(\frac{a c^\prime}{a^\prime} : x) = 1$. Indeed, 
since $x \in \{a n+b \}$, $(a : x) = 1$ and since
$x \in \{c^\prime n+d^\prime \}$, $(c^\prime, x) = 1$,
hence $(a c^\prime : x) = 1$ and moreover
$(\frac{a c^\prime}{a^\prime} : x) = 1$.
Therefore $A \in \mathcal{B}_G$ and of course $A \subseteq \{a n+b \}$.
From our assumption we know that $A \cap X \not= \emptyset$,
but $A \subseteq \{c^\prime n+d^\prime \}$, 
so $\{c^\prime n+d^\prime \} \cap X \not= \emptyset$.
\end{proof}

\begin{thm}
$\I_G \not\subseteq \I_K$.
\end{thm}	   
\begin{proof}
Define $\mathcal{C} = \{\{an+b\}\in \mathcal{B}_K\ :\ \{an+b\}\subseteq \{2n+1\}\}$. Let $\{C_k\ :\ k\in\N\}$ be an enumeration of $\mathcal{C}$.
We will construct a set $X \in \I_G \setminus \I_K$. For every $k\in\N$ pick $x_k$ such that:
\begin{itemize}
	\item $x_k\in C_k$;
	\item $x_k\in \{2^k n+1\}$.
\end{itemize}
Observe that $C_k \cap \{2^k n+1\}$ is always nonempty (if $C_k = \{a_k n+b_k\}$, then $a_k$ is even and square-free, and $(\frac{a_k}{2},\frac{2^k}{2})=(\frac{a_k}{2},2)=1$, as $\frac{a_k}{2}$ is odd; both $\{a_k n+b_k\}$ and $\{2^k n+1\}$ are subsequences of $\{2n+1\}$, so, by the Chinese remainder theorem, their intersection is nonempty). Let $X = \{x_k\ :\ k\in\N\}$.
Firstly, note that $X \notin \I_K$. Indeed, a set $\{2n+1\}\in\mathcal{B}_K$ has a property that for every $\{cn+d\}\subseteq \{2n+1\}$ with $\{cn+d\}\in \mathcal{B}_K$ (so $\{cn+d\}=C_k$ for some $k\in\N$) we have: $X\cap \{cn+d\} \neq \emptyset$ (as it contains $x_k$).
Now, we will prove that $X \in \I_G$. Take any $\{an+b\}\in\mathcal{B}_G$. We want to show that there exists $V\subseteq \{an+b\}$ with $V\in \T_G$ such that $X\cap V = \emptyset$. 
Assume that $2\nmid a$. Let $V = (\{an+b\} \cap \{4n+3\})\setminus\{x_1\}$. $V$ is open (as a difference of an open set and a closed set) and nonempty (because $(a,4)=1$). Moreover, for every $k\geq 2$ we have that $x_k\in \{4n+1\}$, which is disjoint from $\{4n+3\}$, and hence $X\cap V = \emptyset$.
Now, consider the following cases:
\begin{itemize}
 \item $2\mid a$ and $4\nmid a$. Observe that then $2 \nmid b$ (otherwise, $a$ and $b$ would not be coprime), so $\{an+b\}\subseteq \{2n+1\}$. Again, let $V = (\{an+b\} \cap \{4n+3\})\setminus\{x_1\}$. $V$ is open and nonempty (because $(\frac{a}{2},\frac{4}{2})=(\frac{a}{2},2)=1$, as $\frac{a}{2}$ is odd, and both $\{an+b\}$ and $\{4n+3\}$ are subsequences of $\{2n+1\}$). Moreover, as above, $X\cap V = \emptyset$.
 \item $4\mid a$ and $8\nmid a$. Then, as in the previous case, $\{an+b\}\subseteq \{2n+1\}$. Now, either $b\equiv 1 (\textrm{mod } 4)$ or $b\equiv 3 (\textrm{mod } 4)$. Thus, either $\{an+b\}\subseteq \{4n+1\}$ or $\{an+b\}\subseteq \{4n+3\}$. In the first case, let $V = (\{an+b\} \cap \{8n+5\})\setminus\{x_1, x_2\}$. $V$ is open and nonempty (because $(\frac{a}{4},\frac{8}{4})=(\frac{a}{4},2)=1$, as $\frac{a}{4}$ is odd, and both $\{an+b\}$ and $\{8n+5\}$ are subsequences of $\{4n+1\}$). Moreover, for every $k\geq 3$ we have that $x_k\in \{8n+1\}$, which is disjoint from $\{8n+5\}$, and hence $X\cap V = \emptyset$. In the second case, take $V = \{an+b\} \setminus\{x_1\}$. $V$ is open and nonempty. Moreover, for every $k\geq 2$ we have that $x_k\in \{4n+1\}$, which is disjoint from $\{4n+3\}$, and hence $X\cap V = \emptyset$.
\end{itemize}
Generally, for $m\in\N$:
\begin{itemize}
 \item $2^m\mid a$ and $2^{m+1}\nmid a$. Observe that then $2 \nmid b$ (otherwise, $a$ and $b$ would not be coprime), so $\{an+b\}\subseteq \{2n+1\}$. Now, either $b\equiv 1 (\textrm{mod } 2^m)$ or $b \not\equiv 1 (\textrm{mod } 2^m)$. In the first case, let $V = (\{an+b\} \cap \{2^{m+1}n+2^m+1\})\setminus\{x_1, x_2,\ldots, x_m\}$. $V$ is open and nonempty (because $(\frac{a}{2^m},\frac{2^{m+1}}{2^m})=(\frac{a}{2^m},2)=1$, as $\frac{a}{2^m}$ is odd, and both $\{an+b\}$ and $\{2^{m+1}n+2^m+1\}$ are subsequences of $\{2^m n+1\}$). Moreover, for every $k\geq m+1$ we have that $x_k\in \{2^{m+1}n+1\}$, which is disjoint from $\{2^{m+1}n+2^m+1\}$, and hence $X\cap V = \emptyset$. In the second case, take $V = \{an+b\} \setminus\{x_1x_2,\ldots, x_{m-1}\}$. $V$ is open and nonempty. Moreover, for every $k\geq m$ we have that $x_k\in \{2^m n+1\}$, which is disjoint from $\{2^m n+b\}$, and hence $X\cap V = \emptyset$.
\end{itemize}
\end{proof}

-------------------------------




\color{black}
\section{Marczewski-Burstin representations}

\begin{df}
Let us call an ideal $\I\subseteq \mathcal{P}(\N)$ 
\emph{MB-countably-definable}
(briefly: \emph{MBC}) if there exists 
a countable family $\mathcal{F}\subseteq [\omega]^\omega$
%, $|\mathcal{F}|\leq \aleph_0$  -  zbedne bo juz jest slowko: ''countably''
such that
$$\I = MB(\mathcal{F}) = \{X\subseteq\N\ :\ \forall_{F\in\mathcal{F}} \exists_{G\subseteq F, G\in\mathcal{F}} G\cap X=\emptyset\}.$$
\end{df}
%%% Z reguły wymagamy, by $\mathcal{F}\subseteq [\omega]^\omega$  - wstawilem to wprost do definicji MB
%Jeśli $\mathcal{F}\cap MB(\mathcal{F}) = \emptyset$ (to jest coś \`{a} la "`inner MB representation"')
%to mówimy że jest to inner representation.\\
%\textbf{Uwaga:} Chyba zawsze przecież mamy $\mathcal{F}\cap MB(\mathcal{F}) = \emptyset$.\\
%Jeśli $\I$ jest MBC, to dla wielu ideałów $\J$ (np. sumowalnych, gęstościowych...) mamy $\J\nsubseteq \I$.\\
\textbf{Remark:} $\Fin = MB(\mathcal{F})$, where $\mathcal{F}= \{[n, \infty]\cap\N\ :\ n\in\N\}$, 
hence $\Fin\in MBC$.\\
\textbf{Remark:} $\mathrm{NWD}$ has the $\mathcal{MBC}$ property. Indeed, let $b\colon \N \to \Q$
be a fixed bijection. Define
$\mathcal{F} = \lbrace \lbrace n\in\N\colon b(n) \in (p, q)\rbrace\colon p < q, p, q \in \Q\rbrace$.
One can see that $MB(\mathcal{F}) = \mathrm{NWD}$.
		
    \color{teal}

Przypomnijmy, że zbiór $A$ jest \emph{nigdziegęsty w topologii $\T$}, jeśli:
$$\Forall_{U\in\T\setminus\{\emptyset\}}\ \Exists_{V\in\T\setminus\{\emptyset\},\ V\subseteq U}\ A\cap V = \emptyset.$$
Autorzy \cite{MB} zauważyli, że schemat, według którego otrzymuje się rodzinę zbiorów nigdziegęstych z topologii okazuje się być interesujący nawet wtedy, gdy rodzinę zbiorów otwartych zastąpimy dowolną rodziną niepustych podzbiorów dowolnego zbioru X:

\begin{df}[\cite{MB}] Dla $X\neq\emptyset$ i rodziny $\mathcal{F}\subseteq \mathcal{P}(X)\setminus\{\emptyset\}$:
$$S^0(\mathcal{F}) = \left\{A\subseteq X\ :\ \Forall_{U\in\mathcal{F}}\ \Exists_{V\in\mathcal{F}}\ V\subseteq U\setminus A\right\}$$
jest ideałem, nazywanym \emph{ideałem Marczewskiego}.
\end{df}

W podobny sposób można zdefiniować ciało zbiorów (rodzinę zbiorów zamkniętą na dopełnienia i skończone sumy jej elementów, zawierającą zbiór pusty).

\begin{df}[\cite{MB}] Dla $X\neq\emptyset$ i rodziny $\mathcal{F}\subseteq \mathcal{P}(X)\setminus\{\emptyset\}$:
$$S(\mathcal{F}) = \left\{A\subseteq X\ :\ \Forall_{U\in\mathcal{F}}\ \Exists_{V\in\mathcal{F}}\ (V\subseteq U\setminus A\ \vee\ V\subseteq U\cap A)\right\}$$
jest ciałem, nazywanym \emph{ciałem Marczewskiego}.
\end{df}

W literaturze często można znaleźć inną terminologię -- autorzy \cite{BET} mówią, że "$\mathcal{F}$~jest bazą dla charakteryzacji typu Marczewskiego-Burstina", a autorzy \cite{MB2} nazywają te ideały i ciała "MB-reprezentowalnymi".  

W przypadku, gdy za $\mathcal{F}$ weźmiemy rodzinę wszystkich zbiorów doskonałych w~danej przestrzeni polskiej, $S^0(\mathcal{F})$ i $S(\mathcal{F})$ stanowią dokładnie rodziny klasycznych zbiorów Marczewskiego $(s)$ i $(s^0)$ (zob. \cite{Sz}). Schemat definiujący te zbiory był używany w bardziej ogólnym kontekście np. w \cite{Mo}, \cite{Pa}, \cite{Re}. Wreszcie, niedawno powstał cykl publikacji autorstwa m.in. M. Balcerzaka, A. Bartoszewicza i~K.~Ciesielskiego (np. \cite{MB}, \cite{MB2}, \cite{MB3}), badających te ideały i ciała pod kątem różnych rodzin je generujących.

Zauważmy, że jeżeli za $\mathcal{F}$ weźmiemy rodzinę wszystkich niepustych zbiorów otwartych w danej przestrzeni topologicznej $(X,\T)$, $S^0(\T\setminus\{\emptyset\})$ będzie się składać ze zbiorów nigdziegęstych w tej topologii, natomiast $S(\T\setminus\{\emptyset\})$ będzie rodziną zbiorów z nigdziegęstym brzegiem.
\color{black}

Notice that all our ideals $\I_F$, $\I_G$ i $\I_K$ have the $\mathcal{MBC}$ property.

% jest więc ideałem Marczewskiego. W ramach przyszłej pracy naukowej planuję spróbować rozszerzyć wyniki otrzymane dla %naszych ideałów na klasę ideałów MB-reprezentowal-nych oraz zbadać ciała Marczewskiego dla opisywanych
% topologii na $%\N$, związanych z ciągami arytmetycznymi. W szczególności, zamierzam sprawdzić, 
%czy wszystkie ideały Marczewskiego są %reprezentowane topologicznie, co pozwoliłoby połączyć te~dwa nurty badań.
%Article [CJ] was devoted to extensive studies of topological ideals; the authors considered also an additional requirement stating that an ideal consists of meager sets in some topology. 
%
%
%
%
%
%
%

\begin{thm}
$\I_F$, $\I_G$ and $\I_K$ are $F_{\sigma\delta}$ but not $F_{\sigma}$ ideals.
\end{thm}

\begin{proof}
Let us formulate a little more general lemma
\begin{lem}
Every \emph{MBC} ideal is of type $F_{\sigma\delta}$.
\end{lem}
%@@
\begin{proof}
Suppose that $|\mathcal{F}|\leq\aleph_0$, $\mathcal{F}\subseteq [\omega]^\omega$ is such that $\I = MB(\mathcal{F})$.
Then $X\in MB(\mathcal{F}) \equiv X\in \bigcap_{F\in\mathcal{F}} \bigcup_{G\subseteq F, G\in\mathcal{F}} \{x\in 2^\omega\ :\ x\upharpoonright G = 0\upharpoonright G\}$ 
and this last set is of type $F_{\sigma\delta}$
%Skoro $MB(\mathcal{F}) = \{X\subseteq\N\ :\ \forall_{F\in\mathcal{F}}
%\exists_{G\subseteq F, G\in\mathcal{F}} G\cap X=\emptyset\}$, to 
%Czyli $MB(\mathcal{F})$ jest ideałem typu $F_{\sigma\delta}$.\\
\end{proof}
\end{proof}

\begin{thm}
[Sprawdzić, czy są tu poczynione wszystkie niezbędne założenia do otrzymania tezy]:\\
Suppose that $\mathcal{F}\subseteq [\omega]^\omega$,
$|\mathcal{F}|\leq\aleph_0$. 
Assume also that
$\forall_{F_1, F_2 \in \mathcal{F}} \exists_{G \in \mathcal{F}} G \subseteq F_1 \cap F_2$.
%%% \color{purple}
%%%Pewnie trzeba założyć, że $\forall_{F_1, F_2 \in \mathcal{F}} (F_1 \cap F_2 \neq \emptyset \Rightarrow\exists_{G \subseteq F_1 \cap F_2})
%%%$ (a la baza), w przypadku topologii F, G, K to jest spełnione, więcej, $F_1 \cap F_2$ jest sumą pewnej podrodziny $\mathcal{F_0} 
%%%%\subseteq \mathcal{F}$. 
%%% \color{black}
Moreover, assume that $\mathcal{F}$ has the splitting property. 
%%(tzn. $\forall_ {F \in \mathcal{F}} \exists_{F_1, F_2 
%%\in \mathcal{F}, F_1, F_2 \subseteq F} F_1 \cap F_2 
%%= \emptyset$). 
Then the ideal $\I =  MB(\mathcal{F})$ 
is not an $F_{\sigma}$.
\end{thm}

%Z pracy K. Mazura {\cite[Lemat~1.2]{Mazur}} wiadomo, że $\I$ jest ideałem $F_\sigma$ wtedy i tylko wtedy, gdy $\I= \fin(\phi)=\{A\subseteq \N: \phi(A)<\infty  \}$ dla pewnej półciągłej z dołu podmiary $\phi$.
%Mazur K.: {F_\sigma}-ideals and {\omega_1\omega_1^*}-gaps in the Boolean algebras {\mathcal{P}(\omega)/\mathcal{I}}. Fund. Math. 138, 103–111

\begin{proof}
By the characterization of $F_\sigma$ ideals from \cite{Maz} there exists a lsc submeasure 
$\phi \colon \mathcal{P}(\omega) \to [0, \infty]$, i.e. a function such that
\begin{itemize}
%%% hmm, a co z tym warunkiem ponizej? W surveyu "Ideal convergence" Filipow/Natkaniec
%%% tego warunku tam nie ma - rozumiem ze chodzi o unikniecie sytuacji ze singleton nie
%%% nalezy do idealu, czyz sie nie myle?
%%%\item $\phi(\{n\})<\infty$,
\item $\phi(\omega)>0$,
\item $\phi(A\cup B) \leq \phi(A) + \phi(B)$,
\item $A\subseteq B \Rightarrow\phi(A)\subseteq \phi(B)$,
\item \label{continuity-condition} $\lim_{n\to\infty} \phi(n \cap A) = \phi(A)$.
\end{itemize}
which has the property that 
$\I= \Fin(\phi)=\{A\subseteq \N: \phi(A)<\infty\}$
To prove the previous theorem we need 
the following result, which may be also interesting in itself. 
\end{proof}

\begin{thm}
Let $\mathcal{F}\subseteq [\omega]^\omega$ be a countable family. Assume moreover that 
%%%$\forall_{n\in\N}\forall_{F_1, F_2 \in \mathcal{F}, n \in F_1 \cap F_2  } 
%%%\exists_{G \in \mathcal{F}}
%%%n \in G \subseteq F_1 \cap F_2$.
%%% wyglada na to ze przynajmniej tutaj bedzie potrzebne jednak slabsze zalozenie:
$\forall_{F_1, F_2 \in \mathcal{F}} \exists_{G \in \mathcal{F}} G \subseteq F_1 \cap F_2$.
%% powyzszy wzor ,,wystaje'' poza margines - nie mam wszakze pojecia jak ukrocic jego narowy
Assume also that $\mathcal{F}$ has the splitting property. 
Then $S^0(\mathcal{F})$ is tall.
\end{thm}

\begin{proof}
Let $A \in [\omega]^\omega$ be any set.
Suppose that $A\notin S^0(\mathcal{F})$. 
Then there exists a set $F\in\mathcal{F}$ such that 
$\forall_ {G \in \mathcal{F}, G\subseteq F} G\cap A \neq \emptyset$. 
By the splitting property find a sequence
$(F_n)_{n=0}^\infty \subseteq\mathcal{F}$ such that:
\begin{itemize}
\item[a)] $\forall_ {i\neq j} F_i\cap F_j = \emptyset$,
\item[b)] $\forall_ {i} F_i \subseteq F$.
\end{itemize}
Since $\forall_ {i} F_i \cap A \neq \emptyset$ let us choose 
for each $i$ any element $b_i \in F_i \cap A$ 
and define 
$B:=\{b_i\ :\ i\in\omega\}$. Suppose that $G\in \mathcal{F}$. 
If $G\cap B=\emptyset$ then the proof is finished. If $G\cap B \neq\emptyset$ then let $i_0\in\omega$ 
be such that $b_{i_0}\in G$, so $F_i\cap G \neq\emptyset$, hence there exists 
$H\subseteq F_i\cap G$, $H\in\mathcal{F}$. Then $|H\cap B|\leq 1$ and using the
splitting property for H we obtain that there exists $H^*\subseteq H$, $H^*\in\mathcal{F}$, 
such that $H^* \cap B=\emptyset$. 
Hence $B\notin S^0(\mathcal{F})$

Notice that 
a similar proof shows that if $(F_i)_{i\in\omega}$ 
is a sequence of pairwise disjoint elements from $\mathcal{F}$ and if we
denote by $\calK_(F_i)$ the ideal of sets $A\subseteq \omega$
with the property that 
$A\setminus \bigcup_{i\in\omega} F_i\in\Fin$ and $\forall_{i} A\cap F_i\in\Fin$, then 
$\calK_{(F_i)}\subseteq S^0(\mathcal{F})$.\\
% prawdopodobnie uzasadnienie ponizej jest zbedne?
%\textbf{Dowód:} Niech $F\in\mathcal{F}$. Jeśli $A\cap F=\emptyset$, to koniec dowodu. 
%Załóżmy teraz, że $A\cap F \neq\emptyset$. Wybierzmy $i\in\omega$ takie, że $F\cap F_i \neq\emptyset$. 
%Z założenia istnieje $H\in\mathcal{F}$ takie, że $H\subseteq F\cap F_i$. 
%Wówczas $|H\cap A|<\aleph_0$ (bo $H\cap A\subseteq F_i\cap A$), 
%i znów odpowiednio wiele razy "rozsplittowując" $H$ (de facto możemy zrobić to $\aleph_0$ razy!), 
%znajdujemy $H^*\in\mathcal{F}$, $H^*%\subseteq H$ takie, że $H^* \cap A=\emptyset$, co kończy dowód.
\end{proof}

%%%Wracamy do rozumowania o ideale $F_\sigma$:
\begin{proof} (cont. proof)
Suppose that there exists a lsc submeasure $\phi$ such that
$$S^0(\mathcal{F}) = \Fin(\phi)$$.
Fix a family $\{F_i\}_{i\in\omega}$, $F_i\in\mathcal{F}$, $\forall_{i\neq j} F_i\neq F_j$. 
Thus $\phi(F_i)=\infty$. For each $i\in\omega$ let us choose 
(by the condition \ref{continuity-condition}) 
%czyli zalezenie ze $\lim_{n\to\infty} 
%\phi(n \cap A) = \mu(A)$) 
a sequence of finite sets 
$A_i\subseteq F_i$ such that $\phi(A_i)>i$. Define: $A=\bigcup_{i\in\omega} A_i$. From the previous 
result it follows that $A\in S^0(\mathcal{F})$. 
But on the other hand $\forall_{i\in\omega} A_i \subseteq A$, so $i<\phi(A_i)\leq \phi(A)$, 
so $\phi(A)=\infty$, which is a contradiction.
\end{proof}

    \color{black}


\section{Topological representations}

We show that our new ideals (on countable set) may be connected to some $\sigma$-ideals in separable metrizable spaces. This connection has been introduced by M. Sabok and J. Zapletal in \cite{Sabok}. 
% $\sigma$-ideals of compact sets in separable metrizable spaces

\begin{df} (\cite{Sabok})
Suppose that $X$ is a separable metrizable space, $D\subseteq X$ -- a dense countable set, and $I$ -- $\sigma$-ideal on $X$ containing all singletons. Then
$$\mathcal{J}_I=\left\{A\subseteq D\ :\ cl(A)\in I\right\}$$
is an ideal on $D$. Given an ideal $\mathcal{I}$ on $\N$ we say that $\mathcal{I}$ \emph{has a topological representation} if there are $I,D,X$ as above, for which $\mathcal{I}$ is isomorphic to $\mathcal{J}_I$ (i.e. there exists a bijection $f\colon \N\to D$ such that $A\in\I \Leftrightarrow f[A]\in\mathcal{J}_I$). In such a case we say that $\mathcal{I}$ \emph{is represented on $X$ by $I$}.
\end{df}

% Obviously, J I depends only on the family of closed sets that belong to I . In principle, J I also depends on the set D, which is equal to \bigcup J I , but we will see (Proposition 2.1) that, up to isomorphism, this definition is independent of the choice of D. The ideals of the form J I have been recently studied in [22] and used in canonization (see [13]) of smooth equivalence relations for σ-ideals generated by closed sets. 

If $\mathcal{I}$ on $\N$ has a topological representation, then it can be represented on the Cantor space $2^\N$ via an identification of $\N$ with the set of rationals in the Cantor space. 
% If J has a topological representation, then it is represented on the Cantor space by a σ-ideal generated by a family of compact nowhere dense sets.

\textcolor{red}{???}





\begin{df} Let $\mathcal{I}$ be an ideal on $\N$.
\begin{enumerate}
\item[(i)] $\mathcal{I}$ is \emph{tall} if any infinite set in $\N$ contains an infinite subset that belongs to $\mathcal{I}$. 
\item[(ii)] $\mathcal{I}$ is \emph{$\omega$-$+$-diagonalizable} if there is a countable family $\{X_n\ :\ n\in\N\}$ of subsets of $\N$ such that for any $A\in \mathcal{I}$ there is $n\in\N$ with $A\cap X_n=\emptyset$.
\item[(iii)] $\mathcal{I}$ is \emph{countably separated} if there is a countable family $\{X_n\ :\ n\in\N\}$ of subsets of $\N$ such that for any $A\in \mathcal{I}$ and $B\notin \mathcal{I}$ there is $n\in\N$ with $A\cap X_n=\emptyset$ and $B\cap X_n\notin \mathcal{I}$. % In such a case we say that the family $\{X_n:\ n\in\N\}$ \emph{separates} $\mathcal{I}$.
\item[(iv)] $\mathcal{I}$ is \emph{weakly selective} if for every partition $(X_n)_{n\in\N}$ of $\N$ such that $X_i\in\I$ for $i\geq 2$ and $\bigcup_{i\geq 2} X_i \notin\I$ there exists a selector of the partition $(X_n)$ which does not belong to $\I$.
\end{enumerate}
\end{df}


--------------------------------------------

\begin{thm}
Ideals $\I_F$, $\I_G$ and $\I_K$ are $\omega$-$+$-diagonalizable.
\end{thm}

\begin{proof}
Let the base $\mathcal{B}_F$ of the Furstenberg's topology be the required countable family $\{X_n\ :\ n\in\N\}$ of subsets of $\N$. Take any $A\in \I_F$. We know that for any $X_m \in \mathcal{B}_F$ there exists $X_k \in \mathcal{B}_F$ with $X_k \subseteq X_m$, such that $A\cap X_k = \emptyset$. Thus, $X_k$ satisfies the condition from the definition of $\omega$-$+$-diagonalizability.\\
The proof goes analogously for $\I_G$ and $\I_K$.
\end{proof}


\begin{thm}
Ideals $\I_F$, $\I_G$ and $\I_K$ are countably separated.
\end{thm}

\begin{proof}
Let the base $\mathcal{B}_F$ of the Furstenberg's topology be the required countable family $\{X_n\ :\ n\in\N\}$ of subsets of $\N$. Take any $A\in \I_F$ and $B\notin \I_F$. We know that for any $X_{m'} \in \mathcal{B}_F$ there exists $X_{k'} \in \mathcal{B}_F$ with $X_{k'} \subseteq X_{m'}$, such that $A\cap X_{k'} = \emptyset$. Moreover, there exists $X_{m''} \in \mathcal{B}_F$ such that for any $X_{k''} \in \mathcal{B}_F$ with $X_{k''} \subseteq X_{m''}$, we have $B\cap X_{k''} \neq \emptyset$. 
We need to show that there is $n\in\N$, for which $A\cap X_n=\emptyset$ and $B\cap X_n\notin \I_F$. Note that $B\cap X_n\notin \I_F$ if and only if there exists $X_{m} \in \mathcal{B}_F$ such that for any $X_{k} \in \mathcal{B}_F$ with $X_{k} \subseteq X_{m}$, we have $B\cap X_n\cap X_{k} \neq \emptyset$. 
Now, take $X_{m''} \in \mathcal{B}_F$. As $A\in\I_F$, there exists $X_{k'} \in \mathcal{B}_F$ with $X_{k'} \subseteq X_{m''}$, such that $A\cap X_{k'} = \emptyset$. Put $X_n = X_{k'}$. It is clear that $A\cap X_n = \emptyset$. Let us check that the second condition is also satisfied. For $X_{m} \in \mathcal{B}_F$ we put $X_n$, and we set any $X_{k} \in \mathcal{B}_F$ with $X_{k} \subseteq X_n$. Then, $B\cap X_n\cap X_{k} = B\cap X_{k}\neq \emptyset$ as $X_k \subseteq X_n = X_{k'} \subseteq X_{m''}$ (and we use the assumption that $B\notin \I_F$).\\
The proof goes analogously for $\I_G$ and $\I_K$.
\end{proof}

A. Kwela and P. Zakrzewski showed in \cite[Proposition 4.3]{KwelaZak} that every countably separated ideal on $\N$ is weakly selective -- so we get an immediate corollary:

\begin{cor}
Ideals $\I_F$, $\I_G$ and $\I_K$ are weakly selective.
\end{cor}

Let us formulate the following more general theorem:

%\begin{remark} (wszystkie powyższe rozumowania działają dla wszystkich %MB z przeliczalną rodziną $\mathcal{F}$ - nigdzie nie korzystamy z c. %%arytm.) \end{remark}
\begin{thm}
Suppose that $\I \subseteq P(\N)$ is an ideal. Then
\begin{itemize}
\item If $\I$ is \emph{MBC} then $\I$ is 
countably separated;
\item If $\I$ is countably separated then there 
exists a \emph{MBC} ideal $\J$ such that $\I \subseteq \J$.
\end{itemize}
\end{thm}

\begin{proof}
Suppose that $\I = \emph{MB}(\calF)$ for some countable
family $\calF \subseteq [\omega]^\omega$.
Let us define $(X_n) = \calF$. Suppose that $A \in \I$
and $B \not\in \I$. Then there exists $F \in \calF$
such that $\forall_{G\in\calF} G \subseteq F \implies G \cap B \not= \emptyset$.
Notice that we have moreover that $\forall_{G\in\calF} G \subseteq F \implies G \cap B \not\in\I$.
Indeed, otherwise we would have that $G\cap B \in \I$ so
for some $H\in\calF$, $H\subseteq G$ and $H\cap B = \emptyset$,
which is impossible. 
Next, there exists $X_n \in \calF$, $X_n \subseteq F$ such 
that $X_n \cap A = \emptyset$. Moreover
$X_n \cap B\not \in \I$ which finishes the proof.

Suppose that
$\{X_n\ :\ n\in\N\}$ is such that for any $A\in \I$ and $B\notin \I$ 
there is $n\in\N$ with $A\cap X_n=\emptyset$ and $B\cap X_n\notin \I$. 

Let us define
\[
\calF = \{
  \cap_{i=1}^k X_{n_i}\colon \cap_{i=1}^k X_{n_i} \not\in \I
	\}.
\]	   
Let us check that $\I \subseteq \emph{MB}(\calF)$.
Suppose that $A\in\I$ and
take any $F\in\calF$, then 
$F = \cap_{i=1}^k X_{n_i}$ for some $n_i\in\N$.
Since $F\not\in\I$ by the assumption about $\I$
we can find $X_n$ such that $A\cap X_n=\emptyset$ and $F\cap X_n\not\in \I$. 
Then $G = \cap_{i=1}^k X_{n_i} \cap X_n\not\in\calF$,
$G\subseteq F$ and $G \cap A = \emptyset$.
\end{proof}

		\color{purple}

%We show that:
%\begin{thm}
%Ideals $\I_F$, $\I_G$ and $\I_K$ are $\omega$-$+$-diagonalizable.
%\end{thm}

W pracy \cite{Adas} autorzy udowodnili następującą charakteryzację:
\begin{thm}[\cite{Adas}]
An ideal on a countable set has a topological representation if and only if it is tall and countably separated.
\end{thm}

Sprawdziliśmy więc, że:
\begin{thm}
Ideals $\I_F$, $\I_G$ and $\I_K$ are tall and countably separated.
\end{thm}

\begin{cor}
Ideals $\I_F$, $\I_G$ and $\I_K$ have a topological representation.
\end{cor}

W pracy \cite{Sabok} M. Sabok i J. Zapletal postawili następującą hipotezę:
\begin{conj}[\cite{Sabok}]
An ideal on a countable set has a topological representation if and only if it is tall, $F_{\sigma\delta}$ and weakly selective. 
\end{conj}


Pokazali również, że implikacja w prawą stronę jest prawdziwa, więc pytanie dotyczy tylko implikacji w lewą stronę.

%\begin{cor}
%Ideały $\I_F$, $\I_G$ oraz $\I_K$ są słabo selektywne.
%\end{cor}

\color{purple}
Notice that our ideals only provide an example for the conjecture - it does not prove or disprove anything... 
\color{black}

Nasze ideały stanowią zatem przykład potwierdzający hipotezę M. Saboka i J. Zapletala.\\


%\begin{thm}[M. Kwela, A. Nowik]
%Ideals $\I_F$, $\I_G$ and $\I_K$ are tall and $\omega$-$+$-diagonalizable.
%\end{thm}

%\begin{conj}[M. Sabok, J. Zapletal]
%An ideal on a countable set has a topological representation if and only if it is tall and $\omega$-$+$-diagonalizable.
%\end{conj}
%% Borel?.........

%\begin{thm}[A. Kwela, M. Sabok, \cite{Adas}]
%An ideal on a countable set has a topological representation if and only if it is tall and countably separated.
%\end{thm}

%\begin{thm}[M. Kwela, A. Nowik]
%Ideals $\I_F$, $\I_G$ and $\I_K$ are tall and countably separated.
%\end{thm}

%\begin{cor}
%Ideals $\I_F$, $\I_G$ and $\I_K$ have a topological representation.
%\end{cor}

%\color{purple}
%Notice that our ideals only provide an example for the conjecture - it does not prove or disprove anything... 
%\color{black}


So far, only two examples of ideals with topological representations have been studied, namely:
$$\textrm{NWD}=\left\{A\subseteq\mathbb{Q}\cap [0,1]\ :\ cl(A) \textrm{ is meager}\right\},$$
$$\textrm{NULL}=\left\{A\subseteq\mathbb{Q}\cap [0,1]\ :\ cl(A) \textrm{ is of Lebesgue measure zero}\right\},$$
which are the ideals represented on $[0,1]$ by $\sigma$-ideals of meager sets and sets of Lebesgue measure zero, respectively. In \cite{FS} I. Farah and S. Solecki proved that $\textrm{NWD}$ and $\textrm{NULL}$ are non-isomorphic.
% In 2003

\begin{problem}
Are $\I_F$, $\I_G$, $\I_K$, $\textrm{NWD}$ and $\textrm{NULL}$ pairwise non-isomorphic?
\end{problem}


\begin{prop}
$\I_F$ and $\textrm{NWD}(\Q)$ are isomorphic.
\end{prop}
\begin{proof}
Firstly, note that $\textrm{NWD}(\Q)$ can also be defined as:
$$\textrm{NWD}(\Q)=\left\{A\subseteq\mathbb{Q}\cap [0,1]\ :\ A \textrm{ is nowhere dense}\right\}.$$
Clearly, $\N$ with Furstenberg's topology is homeomorphic to $\Q$ (with the natural topology), due to a 1920 theorem of Sierpi\'nski, which characterizes the rationals as the unique countable metrizable space without isolated points. Obviously, $\Q$ is homeomorphic to $\Q\cap (0,1)$. Therefore, $\I_F$ is isomorphic to the ideal $\I = \left\{A\subseteq\Q\cap (0,1)\ :\ A \textrm{ is nowhere dense}\right\}$. It is not hard to observe that $\I$ is isomorphic to $\textrm{NWD}(\Q)$. Set any infinite $A\in\I$ and define a function $f : \Q\cap (0,1) \to \Q\cap [0,1]$ such that $f\upharpoonright A$ is an arbitrary bijection between $A$ and $A\cup\{0,1\}$, and $f\upharpoonright A^C$ is an identity map on $A^C$.
Then $f$ is an isomorphism between the ideals $\I$ and $\textrm{NWD}(\Q)$.
\end{proof}



    \color{purple}
		
\section{Własność $\finbw$ i rozszerzalność do ideałów sumowalnych}

\textbf{\underline{WNIOSKI:}} 
Można bardzo łatwo indukcyjnie skonstruować $A\in \I_{\frac{1}{n}}$ taki, że $A\notin \I_F\cup\I_G\cup\I_K$ ($A$ ma "`haczyć"' wszystkie ciągi arytmetyczne). Zatem $\I_{\frac{1}{n}} \nsubseteq \I_F$. Przykład Primes dowodzi, że $\I_F \nsubseteq \I_{\frac{1}{n}}$. Tymczasem $\{2n+2\}\notin\I_{\frac{1}{n}}$, więc ten przykład dowodzi, że $\I_G \nsubseteq \I_{\frac{1}{n}}$ no i że $\I_K \nsubseteq \I_{\frac{1}{n}}$.

\begin{problem}
Które z ideałów $\I_F$, $\I_G$, $\I_K$ są notFinBW?
\end{problem}

------------------------
\color{teal}

Szeroko badaną tematyką w ostatnich czasach jest zbieżność ideałowa. Równoważne jej pojęcie zbieżności filtrowej pojawiło się już w 1937 roku u H. Cartana (\cite{cartan}), a później zajmowali się nią m.in. M. Laczkovich, I. Recław czy W. Wilczyński.  
\begin{df}
Niech $\I$ będzie ideałem na $\N$, zaś $X$ -- przestrzenią Hausdorffa. Mówimy, że ciąg $(x_n)_{n\in\N}$ elementów z $X$ jest $\I$-zbieżny do $x\in X$, jeżeli dla dowolnego otoczenia otwartego $U$ punktu $x$, $\{n\in \N\ :\ x_n\not\in U  \}\in\I$. %W dalszej części projektu będziemy zakładać, że wszystkie wspominane przestrzenie są Hausdorffa.
\end{df}
Kiedy $\I=\Fin$ (ideał wszystkich zbiorów skończonych), otrzymujemy zwykłą zbieżność ciągów.

%Pojęcie zbieżności ideałowej w obecnej formie zostało wprowadzone w \cite{KSW} przez Kostyrko, {\v{S}}al{\'a}ta i Wilczy{\'n}skiego kilkanaście lat temu, chociaż równoważne mu pojęcie zbieżności filtrowej pojawiło się już w 1937 roku u Cartana \cite{cartan}. 

\color{black}
\section{FinBW property}

In the article \cite{H1} R. Filipów, N. Mrożek, I. Recław and P. Szuca 
defined the Bolzano-Weierstrass property (abbreviation: BW) 
for ideals. This property is a generalization
of well known classical Bolzano-Weierstrass theorem.

\begin{df}[\cite{H1}]
We say that an ideal $\I \subseteq P(\N)$ has:
\begin{enumerate}
\item \emph{the $\bw$ property}, 
iff for each bounded sequence $(x_n)_{n\in\N}$ 
of real numbers there exists $A\notin\I$ 
such that the subsequence $(x_n)_{n\in A}$ is $\I$-convergent;
\item \emph{the $\finbw$ property}, iff for every bounded sequence
$(x_n)_{n\in\N}$ of real numbers there exists $A\notin\I$ 
such that the subsequence $(x_n)_{n\in A}$ is $\Fin$-convergent.
\end{enumerate}
\end{df}
It is easy to see that $\finbw$ implies $\bw$.

In \cite{H1} the authors proved the following ''tree-like'' 
characterization of $\bw$ ideals:
\begin{prop}[\cite{H1}]
An ideal $\I$ is $\bw$ iff for any binary tree $\mathbb{T}$ of high $\omega$ 
such that each level of $\mathbb{T}$ is a partition of $\N$, 
there exists an infinite branch $\{B_n \ :\ n\in \N\}$ of the tree
$\mathbb{T}$ and a set $A\notin\I$ such that $A\setminus B_n \in\I$ for each $n$.
\end{prop}

Also, the authors \cite{BFMS11} proved 
the following characterization of ideals with the property $\finbw$:
\begin{thm}[\cite{BFMS11}]
\label{tree-fin-bw}
An ideal $\I$ does not have the $\finbw$ property iff there exists a 
sequence $(\mathcal{P}_n)$ of finite partitions of $\N$ such that 
$\mathcal{P}_n \sqsubseteq \mathcal{P}_{n + 1}$
(i.e. $\mathcal{P}_n$ is coarser than $\mathcal{P}_{n + 1}$)
and such that if $(A_n)$ is a decreasing sequence 
of sets with the property that $A_n \in \mathcal{P}_n$ for each $n$, 
and a set $Z\subseteq\N$ is such that $|Z\setminus A_n| < \aleph_0$ 
for each $n$ then $Z\in\I$.
\end{thm}

For an arbitrary partition $\mathcal{P}$ of $\N$ 
let us denote \\$H^{*}(\mathcal{P}) = \{A\subseteq\N\ :\ \Exists_{B\in\mathcal{P}} A\subseteq^{*} B\}$, 
where $A\subseteq^{*} B$ is the standard relation of ''almost inclusion'', i.e. 
$A\setminus B$ is a finite set.

We show that the above characterization is equivalent to the following:

\begin{prop}
Ideal $\I$ does not have $\finbw$ property if and only if there exists a sequence $(\mathcal{P}_n)$ of finite partitions of $\N$, such that: 
$$\bigcap_{n\in\N} H^{*}(\mathcal{P}_n)\subseteq\I$$.
\end{prop}

\begin{proof}
Let us assume that $\I$ does not have the $\finbw$ property and let
$(\mathcal{P}_n)$ be a suitable sequence of finite partitions of $\N$
such that $\mathcal{P}_n \prec \mathcal{P}_{n + 1}$ (by Theorem \ref{tree-fin-bw}). 
Let $Z \in \cap_{n\in\N} H^{*}(\mathcal{P}_n)$. For each $n\in\N$ choose $A_n\in \mathcal{P}_n$
such that $Z \subseteq^* A_n$ (notice that such $A_n$ is unique by virtue of fact that  
$\mathcal{P}_n$ is a partition). Then $A_{n+1} \subseteq A_n$. 
Indeed, suppose that it is not true.
Then $A_{n+1} \cap A_n = \emptyset$, which is a contradiction with $Z \subseteq^* A_n$ i $Z \subseteq^* A_{n + 1}$. 
Hence $Z \in \I$, so $\cap_{n\in\N} H^{*}(\mathcal{P}_n) \subseteq \I$.

On the other hand, suppose that $(\mathcal{P}_n)$ is a sequence 
of finite partitions such that $\cap_{n\in\N} H^{*}(\mathcal{P}_n) \subseteq \I$. 
Define $\mathcal{R}_n = \bigsqcup_{i=1}^n \mathcal{P}_n$. 
Suppose that $Z$ is such that there exist $A_n \in R_n$ such that $A_{n+1} \subseteq A_n$ and such that 
$\forall_{n\in\N} Z \subseteq^* A_n$. Then for each $n\in\N$ there exists $B_n \in \mathcal{P}_n$ 
such that $A_n \subseteq B_n$ and therefore for each $n\in\N$ we have $Z \in H^{*}(\mathcal{P}_n)$, 
hence by our assumption we obtain $Z\in\I$.
\end{proof}

%Przy użyciu tej charakteryzacji uzyskaliśmy interesujący wynik dla ideału Furstenberga $\I_F$:
With the use of this characterization we have obtained an interesting result for Furstenberg's, Golomb's and Kirch's ideals: 

\begin{thm}
Ideals $\I_F$, $\I_G$ and $\I_K$ do not have $\finbw$ property.
\end{thm}

\begin{proof}
%%(zmienić oznaczenia...)\\ - czemu? chyba wszystko jest OK...?
For $k\in\N$ 
%%%(and $n=0,1,2,\ldots$ ... i.e. $n\in\N_0$), 
let the sequence $(\mathcal{P}_k)$ of finite partitions of $\N$ be defined as follows:\\
$$\mathcal{P}_1 := \{\{n+1\colon n = 0,1,2,\ldots\}$$
$$\mathcal{P}_2 := \{\{2n+1\colon n = 0,1,2,\ldots\},
 \{2n+2\colon n = 0,1,2,\ldots\}\}$$
$$\mathcal{P}_3 := \{\{3n+1\colon n = 0,1,2,\ldots\},
 \{3n+2\colon n = 0,1,2,\ldots\}, \{3n+3\colon n = 0,1,2,\ldots\}\}$$
$$\vdots$$
$$\mathcal{P}_k := \{\{kn+1\}, \{kn+2\}, \ldots, \{kn+k\}\}$$
$$\vdots$$
%(Each $\mathcal{P}_k$ is a partition of $\N$ into the equivalence classes of
% the relation modulo k...)\\
Then, for every $k\in\N$, 
$$H^{*}(\mathcal{P}_k)= \{A\subseteq\N \ :\ \Exists_{b_k\in\{1,2,\ldots,k\}} A\subseteq^* \{kn+b_k\}\}.$$
Thus,
$$\bigcap_{k\in\N} H^{*}(\mathcal{P}_k)= \{A\subseteq\N \ :\ \Forall_{k\in\N}\Exists_{b_k\in\{1,2,\ldots,k\}} A\subseteq^* \{kn+b_k\}\}.$$
%(Obserwacja: Należy tu np. zbiór $\{n!\}$ -- dla dowolnego $k$ od pewnego miejsca (od $k$-tego wyrazu) wszystkie $n!$ są podzielne przez $k$, czyli $\{n!\}\subseteq^* \{kn+k\}$).

Let us check that $\bigcap_{k\in\N} H^{*}(\mathcal{P}_k)\subseteq\I_F$. 
Choose any set $A\subseteq\N$ such that 
$$\Forall_{k\in\N}\Exists_{b_k\in\{1,2,\ldots,k\}} A\subseteq^* \{kn+b_k\}\}.$$ 
We will show that $A$ is nowhere dense in the Furstenberg 
topology, i.e. we want to show that for every $\{an+b\}\in \mathcal{B}_F$ there exists $\{cn+d\}\subseteq \{an+b\}$ with $\{cn+d\}\in \mathcal{B}_F$ such that $A\cap \{cn+d\} = \emptyset$. Fix $\{an+b\}\in \mathcal{B}_F$ (we then know that $b\leq a$). Then there exists $b_a\in\{1,2,\ldots,a\}$ 
such that $A\subseteq^* \{an+b_a\}$. Let us consider two cases:
\begin{itemize}
	\item[(i)] If $b_a\neq b$ then $\{an+b\} \cap \{an+b_a\} = \emptyset$, so $\{an+b\}\cap A \in\Fin$. Let  $an_{\textrm{max}}+b=\max(\{an+b\}\cap A)$. Then $\{an+an_{\max}+b+a\} \cap A = \emptyset$, but $\{an+an_{\max}+b+a\}\notin \mathcal{B}_F$. Since $an_{\max}+b+a \leq an_{\max}+a+a = (n_{\max}+2)a$, as $\{cn+d\}$ put: $\{cn+d\} = \{(n_{\max}+2)an+an_{\max}+b+a\}\in \mathcal{B}_F$. Then $\{cn+d\}\subseteq \{an+b\}$ and $A\cap \{cn+d\} = \emptyset$.
	\item[(ii)] If $b_a = b$ then $A\subseteq^* \{an+b\}$. Notice that $\{an+b\}=\{2an+b\}\cup \{2an+b+a\}$ (it follows from the ''splitting property'' of $\mathcal{B}_F$). 
By our assumption $b_{2a}\leq 2a$ such that $A\subseteq^* \{2an+b_{2a}\}$. Hence if $b_{2a} \neq b$, and 
we do like in the case (i). If however $b_{2a}=b$, then $b_{2a}\neq b+a$ and again we do like in the case (i).
\end{itemize}

\end{proof}

%Aktualnie pracuję nad rozszerzeniem tego wyniku również na ideały $\I_G$ oraz $\I_K$.\\
%\begin{problem}
%Które z ideałów $\I_F$, $\I_G$, $\I_K$ są notFinBW?
%\end{problem}

Za K. Mazurem (\cite{Maz}), wprowadźmy teraz pojęcie ideału sumowalnego:
\begin{df}[\cite{Maz}]
Ideał $\I$ nazywamy \emph{ideałem sumowalnym}, gdy istnieje szereg rozbieżny $\sum{a_n}$ liczb nieujemnych taki, że $\I=\{A\subseteq \N\ :\ \sum_{n\in A} a_n <\infty\}$.
%Kiedy mamy funkcję $f:\N\rightarrow \R_+$ taką że $\sum_{n\in\N} f(n)=\infty$, ideał $\I_f=\{A\subseteq \N: \sum_{n\in A} f(n)<\infty  \}$ nazywamy \emph{ideałem sumowalnym}.
\end{df}
Najbardziej znanym przykładem ideału sumowalnego jest ideał: 
$$\I_{\frac{1}{n}} = \{A\subseteq \N\ :\ \sum_{n\in A} \frac{1}{n} <\infty\}.$$

Przez wielu matematyków (np. \cite{Au}, \cite{FreedSem} czy, niedawno, \cite{Klinga}) była badana własność \emph{rozszerzalności do ideałów sumowalnych}. Wiąże się ona z twierdzeniem Riemanna o szeregach warunkowo zbieżnych (tj. twierdzeniem mówiącym, że w dowolnym szeregu warunkowo zbieżnym można tak przestawić wyrazy, żeby dostać dowolną ustaloną liczbę jako sumę tego szeregu). W pracy \cite{W} Wilczyński wzmocnił wynik Riemanna, pokazując, że można przestawiać tylko wyrazy, których indeksy tworzą zbiór o asymptotycznej gęstości zero i postawił problem o podanie charakteryzacji wszystkich ideałów mających analogiczną własność, tzn. ideałów $\I$ takich, że dla dowolnego szeregu warunkowo zbieżnego $\sum_n a_n$ i dla i dla dowolnego $r\in\R$ istnieje permutacja $\sigma \colon \N\to\N$ taka, że $\sum_n a_{\sigma(n)} = r$ oraz $\{n\ :\ \sigma(n)\neq n\}\in\I$. Mówimy, że ideały takie mają \emph{własność Riemanna}.

W pracy \cite{H3} R. Filipów i P. Szuca rozwiązali ten problem, dowodząc, że:
\begin{thm}[\cite{H3}]
Ideał ma własność Riemanna wtedy i tylko wtedy, gdy nie rozszerza się do ideału sumowalnego.
\end{thm}

Rozpatrując przykłady wspomniane w podrozdziale 2, można łatwo sprawdzić, że ideały $\I_F$, $\I_G$ i $\I_K$ nie zawierają się w ideale sumowalnym $\I_{\frac{1}{n}}$: ponieważ ani zbiór liczb pierwszych $\mathbb{P}$, ani zbiór liczb parzystych $\{2n+2\}$ nie należą do $\I_{\frac{1}{n}}$, przykład $\mathbb{P}$ dowodzi, że $\I_F \nsubseteq \I_{\frac{1}{n}}$, natomiast przykład $\{2n+2\}$ dowodzi, że $\I_G \nsubseteq \I_{\frac{1}{n}}$ i że $\I_K \nsubseteq \I_{\frac{1}{n}}$. Może to wskazywać na to, że żaden z tych ideałów nie rozszerza się do ideałów sumowalnych.

Wynik z pracy \cite{H1} prowadzi do potwierdzenia tych przypuszczeń.
\begin{prop}[\cite{H1}]
Żaden ideał bez własności $\finbw$ nie rozszerza się do ideału sumowalnego.
\end{prop}

Otrzymujemy więc natychmiastowy wniosek:
\begin{cor}
Ideał $\I_F$ nie rozszerza się do ideału sumowalnego (ma zatem własność Riemanna).
\end{cor}




    \color{black}


% fact/example/prop....?
% porównanie z W, I_d?...





%%%%%%%%%%%%%%%%%%%%%%%%%%%%%%%%%%%%%%%%%%
% do zbadania: 
%  homogeneity and K-uniformity
%  inne topologie
%%%%%%%%%%%%%%%%%%%%%%%%%%%%%%%%%%%%%%%%%%






\begin{thebibliography}{abc}

\bibitem{Au}
Auerbach H., \emph{$\ddot{\textrm{U}}$ber die Vorzeichenverteilung in unendlichen Reihen.},
Studia Math. {\bf 2} (1930) 228--230.

\bibitem{MB}
Balcerzak M., Bartoszewicz A., Rzepecka J., Wroński S., \emph{Marczewski fields and ideals},
Real Anal. Exchange {\bf 26}(2) (2001) 703--715.

\bibitem{MB2}
Balcerzak M., Bartoszewicz A., Ciesielski K., \emph{Algebras with inner MB-representation},
Real Anal. Exchange {\bf 29}(1) (2004) 265--274.

\bibitem{MB3}
Balcerzak M., Bartoszewicz A., Ciesielski K., \emph{On Marczewski-Burstin representations of certain algebras of sets},
Real Anal. Exchange {\bf 26} (2001) 581--592.

\bibitem{BFMS11}
Barbarski P., Filipów R., Mrożek N., Szuca P., \emph{Uniform density $u$ and $\I_u$-convergence on a big set},
Math. Commun. {\bf 16}(1) (2011) 125--130.

\bibitem{BET}
Brown J.B., Elalaoui-Talibi H., \emph{Marczewski-Burstin-like characterizations of $\sigma$-algebras, ideals, and measurable functions},
Colloq. Math. {\bf 82} (1991) 277--286.

\bibitem{B}
Brown M., \emph{A countable connected Hausdorff space},
In: Cohen L.M., The April Meeting in New York, Bull. Amer. Math. Soc. {\bf 59}(4) (1953) 367.

\bibitem{cartan}
Cartan H., \emph{Filtres et ultrafiltres},
C. R. Acad. Sci. Paris {\bf 205} (1937) 777--779.

\bibitem{FS}
Farah I., Solecki S., \emph{Two $F_{\sigma\delta}$ ideals},
Proc. Amer. Math. Soc. {\bf 131}(6) (2003) 1971--1975.

\bibitem{H1}
Filipów R., Mrożek N., Recław I., Szuca P., \emph{Ideal convergence of bounded sequences},
J. Symbolic Logic {\bf 72}(2) (2007) 501--512.

\bibitem{H3}
Filipów R., Szuca P., \emph{Rearrangement of conditionally convergent series on a small set},
J. Math. Anal. Appl. {\bf 362}(1) (2010) 64--71.

\bibitem{F}
Furstenberg H., \emph{On the infinitude of primes},
Amer. Math. Monthly {\bf 62}(5) (1955) 353.

\bibitem{G}
Golomb S.W., \emph{A connected topology for the integers},
Amer. Math. Monthly {\bf 66}(8) (1959) 663--665.

\bibitem{K}
Kirch A.M., \emph{A countable, connected, locally connected Hausdorff space},
Amer. Math. Monthly {\bf 76}(2) (1969) 169--171.

%\bibitem{Klinga}
%Klinga P., \emph{Rearranging series of vectors on a small set},
%J. Math. Anal. Appl. {\bf 424}(2) (2015) 966--974.

\bibitem{Klinga}
Klinga P., Nowik A., \emph{Extendability to summable ideals},
Acta Math. Hungar. {\bf 152}(1) (2017) 150--160.

\bibitem{Adas}
Kwela A., Sabok M., \emph{Topological representations},
J. Math. Anal. Appl. {\bf 422} (2015) 1434--1446.

\bibitem{KwelaZak}
Kwela A., Zakrzewski P., \emph{Combinatorics of ideals -- selectivity versus density}, unpublished extended version available at:
http://kwela.strony.ug.edu.pl/papers/Combinatorics\underline{ }of\underline{ }ide-als\underline{ }extended.pdf,
last accessed October 31st, 2017.

\bibitem{Maz}
Mazur K., \emph{$F_\sigma$-ideals and $\omega_1\omega_1^*$-gaps in the Boolean algebras $\mathcal{P}(\omega)/\mathcal{I}$},
Fund. Math. {\bf 138}(2) (1991) 103--111.

\bibitem{Mo}
Morgan II J.C., \emph{Point Set Theory},
Marcel Dekker, New York, 1990.

\bibitem{Pa}
Pawlikowski J., \emph{Parametrized Ellentuck theorem},
Topology Appl. {\bf 37} (1990) 65--73.

\bibitem{Re}
Reardon P., \emph{Ramsey, Lebesgue and Marczewski sets and the Baire property},
Fund. Math. {\bf 149} (1996) 191--203.

\bibitem{Sabok} 
Sabok M., Zapletal J., \emph{Forcing  properties  of  ideals  of  closed  sets}, 
J. Symbolic Logic {\bf 76}(3) (2011) 1075--1095.

\bibitem{FreedSem}
Semer J.J., Freedman A.R., \emph{On summing sequences of 0’s and 1’s},
Rocky Mountain J. Math. {\bf 11}(3) (1981) 419--425.

\bibitem{Szczuka1} 
Szczuka P., \emph{Connections between connected topological spaces on the set of positive integers}, 
Cent. Eur. J. Math. {\bf 11}(5) (2013) 876--881.

\bibitem{Szczuka2}
Szczuka P., \emph{Properties of the division topology on the set of positive integers},
Int. J. Number Theory {\bf 12}(3) (2016) 775--785.

\bibitem{Szczuka3}
Szczuka P., \emph{The closures of arithmetic progressions in the common division topology on the set of positive integers},
Cent. Eur. J. Math. {\bf 12}(7) (2014) 1008--1014.

\bibitem{Szczuka4}
Szczuka P., \emph{The connectedness of arithmetic progressions in Furstenberg's, Golomb's and Kirch's topologies},
Demonstratio Math. {\bf 43}(4) (2010) 899--909.

\bibitem{Sz}
Szpilrajn (Marczewski) E., \emph{Sur une classe de fonctions de M. Sierpiński et la classe correspondante d'ensembles},
Fund. Math. {\bf 24} (1935) 17--34.

\bibitem{W}
Wilczyński W., \emph{On Riemann derangement theorem},
Słupskie Prace Matematyczno-Fizyczne {\bf 4} (2007) 79--82.

\end{thebibliography}

\end{document}
